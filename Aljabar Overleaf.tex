% Options for packages loaded elsewhere
\PassOptionsToPackage{unicode}{hyperref}
\PassOptionsToPackage{hyphens}{url}
\documentclass[
]{book}
\usepackage{xcolor}
\usepackage{amsmath,amssymb}
\setcounter{secnumdepth}{-\maxdimen} % remove section numbering
\usepackage{iftex}
\ifPDFTeX
  \usepackage[T1]{fontenc}
  \usepackage[utf8]{inputenc}
  \usepackage{textcomp} % provide euro and other symbols
\else % if luatex or xetex
  \usepackage{unicode-math} % this also loads fontspec
  \defaultfontfeatures{Scale=MatchLowercase}
  \defaultfontfeatures[\rmfamily]{Ligatures=TeX,Scale=1}
\fi
\usepackage{lmodern}
\ifPDFTeX\else
  % xetex/luatex font selection
\fi
% Use upquote if available, for straight quotes in verbatim environments
\IfFileExists{upquote.sty}{\usepackage{upquote}}{}
\IfFileExists{microtype.sty}{% use microtype if available
  \usepackage[]{microtype}
  \UseMicrotypeSet[protrusion]{basicmath} % disable protrusion for tt fonts
}{}
\makeatletter
\@ifundefined{KOMAClassName}{% if non-KOMA class
  \IfFileExists{parskip.sty}{%
    \usepackage{parskip}
  }{% else
    \setlength{\parindent}{0pt}
    \setlength{\parskip}{6pt plus 2pt minus 1pt}}
}{% if KOMA class
  \KOMAoptions{parskip=half}}
\makeatother
\setlength{\emergencystretch}{3em} % prevent overfull lines
\providecommand{\tightlist}{%
  \setlength{\itemsep}{0pt}\setlength{\parskip}{0pt}}
\usepackage{bookmark}
\IfFileExists{xurl.sty}{\usepackage{xurl}}{} % add URL line breaks if available
\urlstyle{same}
\hypersetup{
  hidelinks,
  pdfcreator={LaTeX via pandoc}}

\author{}
\date{}

\begin{document}
\frontmatter

\mainmatter
\chapter{EMT untuk Perhitungan Aljabar}\label{emt-untuk-perhitungan-aljabar}

Pada notebook ini Anda belajar menggunakan EMT untuk melakukan berbagai perhitungan terkait dengan materi atau topik dalam Aljabar. Kegiatan yang harus Anda lakukan adalah sebagai berikut:

\begin{itemize}
\item
  Membaca secara cermat dan teliti notebook ini;
\item
  Menerjemahkan teks bahasa Inggris ke bahasa Indonesia;
\end{itemize}

Mencoba contoh-contoh perhitungan (perintah EMT) dengan cara + meng-ENTER setiap perintah EMT yang ada (pindahkan kursor ke baris + perintah)

\begin{itemize}
\item
  Jika perlu Anda dapat memodifikasi perintah yang ada dan memberikan
\item
  keterangan/penjelasan tambahan terkait hasilnya.
\item
  Menyisipkan baris-baris perintah baru untuk mengerjakan soal-soal
\item
  Aljabar dari file PDF yang saya berikan;
\item
  Memberi catatan hasilnya.
\item
  Jika perlu tuliskan soalnya pada teks notebook (menggunakan format
\item
  LaTeX).
\item
  Gunakan tampilan hasil semua perhitungan yang eksak atau simbolik
\item
  dengan format LaTeX. (Seperti contoh-contoh pada notebook ini.)
\end{itemize}

\section{Contoh pertama}\label{contoh-pertama}

Menyederhanakan bentuk aljabar:

\[6x^{-3}y^5\times -7x^2y^{-9}\]\textgreater\$\&6*x\textsuperscript{(-3)*y}5*-7*x\textsuperscript{2*y}(-10)

\[-\frac{42}{x\,y^5}\]Menjabarkan:

\[(6x^{-3}+y^5)(-7x^2-y^{-9})\]\textgreater\$\&showev('expand((6*x\textsuperscript{(-3)+y}5)*(-7*x\textsuperscript{2-y}(-10))))

\[{\it expand}\left(\left(-\frac{1}{y^{10}}-7\,x^2\right)\,\left(y^5+\frac{6}{x^3}\right)\right)=-7\,x^2\,y^5\frac{1}{y^5}-\frac{6}{x^3\,y^{10}}-\frac{42}{x}\]\#\# Baris Perintah

Baris perintah Euler terdiri dari satu atau beberapa perintah Euler diikuti dengan titik koma ``;'' atau koma ``,''. Titik koma mencegah pencetakan hasil. Koma setelah perintah terakhir dapat dihilangkan.

Baris perintah berikut hanya akan mencetak hasil ekspresi, bukan penugasan atau perintah format.

\textgreater r:=2; h:=4; pi*r\^{}2*h/3

\begin{verbatim}
16.7551608191
\end{verbatim}

Perintah harus dipisahkan dengan kosong. Baris perintah berikut mencetak dua hasilnya.

\textgreater pi*2*r*h, \%+2*pi*r*h // Ingat tanda \% menyatakan hasil perhitungan terakhir sebelumnya

\begin{verbatim}
50.2654824574
100.530964915
\end{verbatim}

Baris perintah dijalankan dalam urutan pengguna menekan return. Jadi Anda mendapatkan nilai baru setiap kali Anda mengeksekusi baris kedua.

\textgreater x := 1;

\textgreater x := cos(x) // nilai cosinus (x dalam radian)

\begin{verbatim}
0.540302305868
\end{verbatim}

Di baris pertama x:= 1. Ini membuat x menjadi suatu nilai numerik (bukan fungsi). Tetapi di baris berikutnya dilkakukan x:=cos(x). Karena x sekarang adalah sebuah nilai numerik, komputer tidak lalgi bisa menemukan variabel atau fungsi x.

\textgreater x := cos(x)

\begin{verbatim}
0.857553215846
\end{verbatim}

Jika dua garis dihubungkan dengan ``\ldots{}'' Kedua baris akan selalu dijalankan secara bersamaan.

\textgreater x := 1.5; \ldots{}\\
\textgreater{} x := (x+2/x)/2, x := (x+2/x)/2, x := (x+2/x)/2,

\begin{verbatim}
1.41666666667
1.41421568627
1.41421356237
\end{verbatim}

Ini juga merupakan cara yang baik untuk menyebarkan perintah panjang ke dua baris atau lebih. Anda dapat menekan Ctrl+Return untuk membagi baris menjadi dua pada posisi kursor saat ini, atau Ctlr+Back untuk menggabungkan baris.

Untuk melipat semua multi-baris, tekan Ctrl+L. Kemudian baris berikutnya hanya akan terlihat, jika salah satunya memiliki fokus. Untuk melipat satu multi-baris, mulailah baris pertama dengan ``\%+''.

\textgreater\%+ x=4+5; \ldots{}\\
\textgreater{} // This line will not be visible once the cursor is off the line

Garis yang dimulai dengan \%\% akan sama sekali tidak terlihat.

\begin{verbatim}
81
\end{verbatim}

Euler mendukung loop dalam baris perintah, selama mereka masuk ke dalam satu baris tunggal atau multi-baris. Dalam program, pembatasan ini tidak berlaku, tentu saja. Untuk informasi lebih lanjut, konsultasikan dengan pengantar berikut.

\textgreater x=1; for i=1 to 5; x := (x+2/x)/2, end; // menghitung akar 2

\begin{verbatim}
1.5
1.41666666667
1.41421568627
1.41421356237
1.41421356237
\end{verbatim}

Bukanlah suatu masalah untuk menggunakan multi-baris. Pastikan baris diakhiri dengan '' \ldots``.

\textgreater x := 1.5; // comments go here before the \ldots{}\\
\textgreater{} repeat xnew:=(x+2/x)/2; until xnew\textasciitilde=x; \ldots{}\\
\textgreater{} x := xnew; \ldots{}\\
\textgreater{} end; \ldots{}\\
\textgreater{} x,

\begin{verbatim}
1.41421356237
\end{verbatim}

Struktur bersyarat juga berfungsi.

\textgreater if E\textsuperscript{pi\textgreater pi}E; then ``Thought so!'', endif;

\begin{verbatim}
Thought so!
\end{verbatim}

Saat Anda menjalankan perintah, kursor dapat berada di posisi mana pun di baris perintah. Anda dapat kembali ke perintah sebelumnya atau melompat ke perintah berikutnya dengan tombol panah. Atau Anda dapat mengklik bagian komentar di atas perintah untuk membuka perintah.

Saat Anda menggerakkan kursor di sepanjang garis, pasangan tanda kurung atau tanda kurung pembuka dan penutup akan disorot. Juga, perhatikan baris status. Setelah tanda kurung pembuka fungsi sqrt(), baris status akan menampilkan teks bantuan untuk fungsi tersebut. Jalankan perintah dengan tombol return.

\textgreater sqrt(sin(10°)/cos(20°))

\begin{verbatim}
0.429875017772
\end{verbatim}

Untuk melihat bantuan untuk perintah terbaru, buka jendela bantuan dengan F1. Di sana, Anda dapat memasukkan teks untuk dicari. Pada baris kosong, bantuan untuk jendela bantuan akan ditampilkan. Anda dapat menekan escape untuk menghapus baris, atau untuk menutup jendela bantuan.

Anda dapat mengklik dua kali pada perintah apa pun untuk membuka bantuan untuk perintah ini. Coba klik dua kali perintah exp di bawah ini di baris perintah.

\textgreater exp(log(2.5))

\begin{verbatim}
2.5
\end{verbatim}

Anda juga dapat menyalin dan menempelkan di Euler. Gunakan Ctrl-C dan Ctrl-V untuk ini. Untuk menandai teks, seret mouse atau gunakan shift bersama dengan tombol kursor apa pun. Selain itu, Anda dapat menyalin tanda kurung yang disorot.

\section{Sintaks Dasar}\label{sintaks-dasar}

Euler tahu fungsi matematika yang biasa. Seperti yang telah Anda lihat di atas, fungsi trigonometri bekerja dalam radian atau derajat. Untuk mengonversi ke derajat, tambahkan simbol derajat (dengan tombol F7) ke nilai, atau gunakan fungsi rad(x). Fungsi akar kuadrat disebut sqrt dalam Euler. Tentu saja, x\^{}(1/2) juga dimungkinkan.

Untuk mengatur variabel, gunakan ``='' atau ``:=''. Demi kejelasan, pengantar ini menggunakan bentuk yang terakhir. Ruang tidak masalah. Tetapi ruang di antara perintah diharapkan.

Beberapa perintah dalam satu baris dipisahkan dengan ``,'' atau ``;''. Titik koma menekan output perintah. Di akhir baris perintah, ``,'' diasumsikan, jika ``;'' hilang.

\textgreater g:=9.81; t:=2.5; 1/2*g*t\^{}2

\begin{verbatim}
30.65625
\end{verbatim}

EMT menggunakan sintaks pemrograman untuk ekspresi. Untuk masuk

lateks: e\^{}2 cdot left( frac\{1\}\{3+4 log(0.6)\}+frac\{1\}\{7\} right)

Anda harus menetapkan tanda kurung yang benar dan menggunakan / untuk pecahan. Perhatikan tanda kurung yang disorot untuk mendapatkan bantuan. Perhatikan bahwa konstanta Euler e bernama E di EMT.

\textgreater E\^{}2*(1/(3+4*log(0.6))+1/7)

\begin{verbatim}
8.77908249441
\end{verbatim}

Untuk menghitung ekspresi rumit seperti

lateks: left(frac\{frac17 + frac18 + 2\}\{frac13 + frac12\}right)\^{}2 pi

Anda harus memasukkannya dalam bentuk baris.

\textgreater((1/7 + 1/8 + 2) / (1/3 + 1/2))\^{}2 * pi

\begin{verbatim}
23.2671801626
\end{verbatim}

Letakkan tanda kurung dengan hati-hati di sekitar sub-ekspresi yang perlu dihitung terlebih dahulu. EMT membantu Anda dengan menyoroti ekspresi yang diselesaikan oleh braket penutup. Anda juga harus memasukkan nama ``pi'' untuk huruf Yunani pi.

Hasil perhitungan ini adalah angka floating point. Secara default dicetak dengan akurasi sekitar 12 digit. Di baris perintah berikut, kita juga mempelajari bagaimana kita dapat merujuk ke hasil sebelumnya dalam baris yang sama.

\textgreater1/3+1/7, fraction \%

\begin{verbatim}
0.47619047619
10/21
\end{verbatim}

Perintah Euler dapat berupa ekspresi atau perintah primitif. Ekspresi dibuat dari operator dan fungsi. Jika perlu, itu harus berisi tanda kurung untuk memaksa urutan eksekusi yang benar. Jika diragukan, memasang braket adalah ide yang bagus. Perhatikan bahwa EMT menunjukkan tanda kurung pembuka dan penutup saat mengedit baris perintah.

\textgreater(cos(pi/4)+1)\textsuperscript{3*(sin(pi/4)+1)}2

\begin{verbatim}
14.4978445072
\end{verbatim}

Operator numerik Euler meliputi

\begin{itemize}
\item
  unary atau operator plus
\item
  unary atau operator minus
\end{itemize}

*, /

. Produk matriks

pangkat a\^{}b untuk positif A atau bilangan bulat b (a**b juga berfungsi)

n! Operator faktorial

dan masih banyak lagi.

Berikut adalah beberapa fungsi yang mungkin Anda perlukan. Masih banyak lagi.

sin, cos, tan, atan, asin, acos, rad, derajat

log, exp, log10, sqrt, logbase

bin, logbin, logfac, mod, lantai, ceil, bulat, abs, tanda

conj, re, im, arg, conj, nyata, kompleks

beta, betai, gamma, gamma kompleks, ellrf, ellf, ellrd, elle

bitand, bitor, bitxor, bitnot

Beberapa perintah memiliki alias, misalnya ln untuk log.

\textgreater ln(E\^{}2), arctan(tan(0.5))

\begin{verbatim}
2
0.5
\end{verbatim}

\textgreater sin(30°)

\begin{verbatim}
0.5
\end{verbatim}

Pastikan untuk menggunakan tanda kurung (tanda kurung bundar), setiap kali ada keraguan tentang urutan eksekusi! Berikut ini tidak sama dengan (2\textsuperscript{3)}4, yang merupakan default untuk 2\textsuperscript{3}4 di EMT (beberapa sistem numerik melakukannya dengan cara lain).

\textgreater2\textsuperscript{3}4, (2\textsuperscript{3)}4, 2\textsuperscript{(3}4)

\begin{verbatim}
2.41785163923e+24
4096
2.41785163923e+24
\end{verbatim}

\section{Bilangan Asli}\label{bilangan-asli}

Tipe data utama dalam Euler adalah bilangan real. Real direpresentasikan dalam format IEEE dengan akurasi sekitar 16 digit desimal.

\textgreater longest 1/3

\begin{verbatim}
     0.3333333333333333 
\end{verbatim}

Representasi ganda internal membutuhkan 8 byte.

\textgreater printdual(1/3)

\begin{verbatim}
1.0101010101010101010101010101010101010101010101010101*2^-2
\end{verbatim}

\textgreater printhex(1/3)

\begin{verbatim}
5.5555555555554*16^-1
\end{verbatim}

\section{Strings}\label{strings}

String dalam Euler didefinisikan dengan ``\ldots{}''.

\textgreater{}``Suatu string bisa memuat apa saja''

\begin{verbatim}
Suatu string bisa memuat apa saja
\end{verbatim}

String dapat digabungkan dengan \textbar{} atau dengan +. Ini juga berfungsi dengan angka, yang dikonversi menjadi string dalam kasus ini.

\textgreater{}``Luas lingkaran dengan jari-jari'' + 2 + '' cm adalah '' + pi*4 + '' cm\^{}2''

\begin{verbatim}
Luas lingkaran dengan jari-jari 2 cm adalah 12.5663706144 cm^2
\end{verbatim}

Fungsi print juga mengonversi angka menjadi string. Ini dapat mengambil sejumlah digit dan sejumlah tempat (0 untuk output padat), dan secara optimal satu unit.

\textgreater{}``Rasio Emas :'' + print((1+sqrt(5))/2,5,0)

\begin{verbatim}
Rasio Emas : 1.61803
\end{verbatim}

Ada string khusus bernama none, yang tidak dicetak. Itu dikembalikan oleh beberapa fungsi, ketika hasilnya tidak masalah. (Ini dikembalikan secara otomatis, jika fungsi tidak memiliki pernyataan return.)

\textgreater none

Untuk mengonversi string menjadi angka, cukup evaluasi saja. Ini juga berfungsi untuk ekspresi (lihat di bawah).

\textgreater{}``1234.5''()

\begin{verbatim}
1234.5
\end{verbatim}

Untuk mendefinisikan vektor string, gunakan notasi vektor {[}\ldots{]}

\textgreater v:={[}``affe'',``charlie'',``bravo''{]}

\begin{verbatim}
affe
charlie
bravo
\end{verbatim}

Vektor string kosong dilambangkan dengan {[}none{]}. Vektor string dapat digabungkan.

\textgreater w:={[}none{]}; w\textbar v\textbar v

\begin{verbatim}
affe
charlie
bravo
affe
charlie
bravo
\end{verbatim}

String dapat berisi karakter Unicode. Secara internal, string ini berisi kode UTF-8. Untuk menghasilkan string seperti itu, gunakan u''\ldots'' dan salah satu entitas HTML.

String Unicode dapat digabungkan seperti string lainnya.

\textgreater u''α = '' + 45 + u''°'' // pdfLaTeX mungkin gagal menampilkan secara benar

\begin{verbatim}
α = 45°
\end{verbatim}

I

Dalam komentar, entitas yang sama seperti a, ß dll dapat digunakan. Ini mungkin alternatif cepat untuk Lateks. (Detail lebih lanjut tentang komentar di bawah).

Ada beberapa fungsi untuk membuat atau menganalisis string unicode. Fungsi strtochar() akan mengenali string Unicode, dan menerjemahkannya dengan benar.

\textgreater v=strtochar(u''Ä is a German letter'')

\begin{verbatim}
[196,  32,  105,  115,  32,  97,  32,  71,  101,  114,  109,  97,  110,
32,  108,  101,  116,  116,  101,  114]
\end{verbatim}

Hasilnya adalah suatu vektor angka Unicode. Fungsi kebalikannya adalah chartoutf().

\textgreater v{[}1{]}=strtochar(u''Ü``){[}1{]}; chartoutf(v)

\begin{verbatim}
Ü is a German letter
\end{verbatim}

Fungsi utf() dapat menerjemahkan string dengan entitas dalam variabel ke dalam string Unicode.

\textgreater s=``We have α=β.''; utf(s) // pdfLaTeX mungkin gagal menampilkan secara benar

\begin{verbatim}
We have α=β.
\end{verbatim}

Dimungkinkan juga untuk menggunakan entitas numerik.

\textgreater u''Ähnliches''

\begin{verbatim}
Ähnliches
\end{verbatim}

\section{Nilai Boolean}\label{nilai-boolean}

Nilai Boolean direpresentasikan dengan 1=benar atau 0=salah dalam Euler. String dapat dibandingkan, seperti angka.

\textgreater2\textless1, ``apel''\textless{}``banana''

\begin{verbatim}
0
1
\end{verbatim}

``and'' adalah operator ``\&\&'' dan ``or'' adalah operator ``\textbar\textbar{}'', seperti dalam bahasa C. (Kata ``dan'' dan ``atau'' hanya dapat digunakan dalam kondisi untuk ``jika''.)

\textgreater2\textless E \&\& E\textless3

\begin{verbatim}
1
\end{verbatim}

Operator Boolean mematuhi aturan bahasa matriks.

\textgreater(1:10)\textgreater5, nonzeros(\%)

\begin{verbatim}
[0,  0,  0,  0,  0,  1,  1,  1,  1,  1]
[6,  7,  8,  9,  10]
\end{verbatim}

Anda dapat menggunakan fungsi nonzeros() untuk mengekstrak elemen tertentu dari vektor. Dalam contoh, kita menggunakan isprime(n) bersyarat.

\textgreater N=2\textbar3:2:99 // N berisi elemen 2 dan bilangan2 ganjil dari 3 s.d. 99

\begin{verbatim}
[2,  3,  5,  7,  9,  11,  13,  15,  17,  19,  21,  23,  25,  27,  29,
31,  33,  35,  37,  39,  41,  43,  45,  47,  49,  51,  53,  55,  57,
59,  61,  63,  65,  67,  69,  71,  73,  75,  77,  79,  81,  83,  85,
87,  89,  91,  93,  95,  97,  99]
\end{verbatim}

\textgreater N{[}nonzeros(isprime(N)){]} //pilih anggota2 N yang prima

\begin{verbatim}
[2,  3,  5,  7,  11,  13,  17,  19,  23,  29,  31,  37,  41,  43,  47,
53,  59,  61,  67,  71,  73,  79,  83,  89,  97]
\end{verbatim}

\section{Format dari Output}\label{format-dari-output}

Format keluaran bawaan dari EMT mencetak 12 digit. Untuk memastikan bahwa kita melihat format bawaan tersebut, kita mengatur ulang formatnya.

\textgreater defformat; pi

\begin{verbatim}
3.14159265359
\end{verbatim}

Secara internal, EMT menggunakan standar IEEE untuk angka ganda dengan sekitar 16 digit desimal. Untuk melihat jumlah digit penuh, gunakan perintah ``longestformat'', atau kami menggunakan operator ``terpanjang'' untuk menampilkan hasilnya dalam format terpanjang.

\textgreater longest pi

\begin{verbatim}
      3.141592653589793 
\end{verbatim}

Berikut adalah representasi heksadesimal internal dari bilangan ganda.

\textgreater printhex(pi)

\begin{verbatim}
3.243F6A8885A30*16^0
\end{verbatim}

Format keluaran dapat diubah secara permanen dengan perintah format.

\textgreater format(12,5); 1/3, pi, sin(1)

\begin{verbatim}
    0.33333 
    3.14159 
    0.84147 
\end{verbatim}

Format bawaanya adalah(12).

\textgreater format(12); 1/3

\begin{verbatim}
0.333333333333
\end{verbatim}

Fungsi seperti ``shortestformat'', ``shortformat'', ``longformat'' bekerja untuk vektor dengan cara berikut.

\textgreater shortestformat; random(3,8)

\begin{verbatim}
  0.66    0.2   0.89   0.28   0.53   0.31   0.44    0.3 
  0.28   0.88   0.27    0.7   0.22   0.45   0.31   0.91 
  0.19   0.46  0.095    0.6   0.43   0.73   0.47   0.32 
\end{verbatim}

Format bawaan untuk skalar adalah format(12). Tapi ini bisa diubah.

\textgreater setscalarformat(5); pi

\begin{verbatim}
3.1416
\end{verbatim}

Fungsi ``longestformat'' juga mengatur format skalar.

\textgreater longestformat; pi

\begin{verbatim}
3.141592653589793
\end{verbatim}

Sebagai referensi, berikut adalah daftar format keluaran yang paling penting.

shortestformat shortformat longformat, longestformat\\
format(length,digits) goodformat(length)\\
fracformat(length)\\
defformat

Akurasi internal EMT adalah sekitar 16 tempat desimal, yang merupakan standar IEEE. Angka disimpan dalam format internal ini.

Tetapi format keluaran EMT dapat diatur dengan cara yang fleksibel.

\textgreater longestformat; pi,

\begin{verbatim}
3.141592653589793
\end{verbatim}

\textgreater format(10,5); pi

\begin{verbatim}
  3.14159 
\end{verbatim}

Setelan bawaannya adalah defformat().

\textgreater defformat; // default

Ada operator pendek yang hanya mencetak satu nilai. Operator ``terpanjang'' akan mencetak semua digit yang valid dari suatu angka.

\textgreater longest pi\^{}2/2

\begin{verbatim}
      4.934802200544679 
\end{verbatim}

Ada operator pendek yang hanya mencetak satu nilai. Operator ``terpanjang'' akan mencetak semua digit yang valid dari suatu angka.

\textgreater fraction 1+1/2+1/3+1/4

\begin{verbatim}
25/12
\end{verbatim}

Karena format internal menggunakan cara biner untuk menyimpan angka, nilai 0,1 tidak akan direpresentasikan dengan tepat. Kesalahan bertambah sedikit, seperti yang Anda lihat dalam perhitungan berikut.

\textgreater longest 0.1+0.1+0.1+0.1+0.1+0.1+0.1+0.1+0.1+0.1-1

\begin{verbatim}
 -1.110223024625157e-16 
\end{verbatim}

Tetapi dengan ``longformat'' default Anda tidak akan melihat hal ini. Untuk kenyamanan, output dari angka yang sangat kecil adalah 0.

\textgreater0.1+0.1+0.1+0.1+0.1+0.1+0.1+0.1+0.1+0.1-1

\begin{verbatim}
0
\end{verbatim}

*Ekspresi

String atau nama dapat digunakan untuk menyimpan ekspresi matematika, yang dapat dievaluasi oleh EMT. Untuk ini, gunakan tanda kurung setelah ekspresi. Jika Anda berniat menggunakan string sebagai ekspresi, gunakan konvensi untuk menamainya ``fx'' atau ``fxy'' dll. Ekspresi lebih diutamakan daripada fungsi.

Variabel global dapat digunakan dalam evaluasi.

\textgreater r:=2; fx:=``pi*r\^{}2''; longest fx()

\begin{verbatim}
      12.56637061435917 
\end{verbatim}

Parameter ditetapkan ke x, y, dan z dalam urutan itu. Parameter tambahan dapat ditambahkan menggunakan parameter yang ditetapkan.

\textgreater fx:=``a*sin(x)\^{}2''; fx(5,a=-1)

\begin{verbatim}
-0.919535764538
\end{verbatim}

Perhatikan bahwa ekspresi akan selalu menggunakan variabel global, bahkan jika ada variabel dalam fungsi dengan nama yang sama. (Jika tidak, evaluasi ekspresi dalam fungsi dapat memiliki hasil yang sangat membingungkan bagi pengguna yang memanggil fungsi tersebut.)

\textgreater at:=4; function f(expr,x,at) := expr(x); \ldots{}\\
\textgreater{} f(``at*x\^{}2'',3,5) // computes 4*3\^{}2 not 5*3\^{}2

\begin{verbatim}
36
\end{verbatim}

Jika Anda ingin menggunakan nilai lain untuk ``at'' daripada nilai global, Anda perlu menambahkan ``at=value''.

\textgreater at:=4; function f(expr,x,a) := expr(x,at=a); \ldots{}\\
\textgreater{} f(``at*x\^{}2'',3,5)

\begin{verbatim}
45
\end{verbatim}

Sebagai referensi, kami berkomentar bahwa koleksi panggilan (dibahas di tempat lain) dapat berisi ekspresi. Jadi kita bisa membuat contoh di atas sebagai berikut.

\textgreater at:=4; function f(expr,x) := expr(x); \ldots{}\\
\textgreater{} f(\{\{``at*x\^{}2'',at=5\}\},3)

\begin{verbatim}
45
\end{verbatim}

Ekspresi dalam x sering digunakan seperti fungsi.

Perhatikan bahwa mendefinisikan fungsi dengan nama yang sama seperti ekspresi simbolis global menghapus variabel ini untuk menghindari kebingungan antara ekspresi simbolis dan fungsi.

\textgreater f \&= 5*x;

\textgreater function f(x) := 6*x;

\textgreater f(2)

\begin{verbatim}
12
\end{verbatim}

Sebagai konvensi, ekspresi simbolik atau numerik harus diberi nama fx, fxy dll. Skema penamaan ini tidak boleh digunakan untuk fungsi.

\textgreater fx \&= diff(x\^{}x,x); \$\&fx

\[x^{x}\,\left(\log x+1\right)\]Bentuk khusus dari ekspresi memungkinkan variabel apa pun sebagai parameter yang tidak disebutkan namanya untuk evaluasi ekspresi, bukan hanya ``x'', ``y'' dll. Untuk ini, mulailah ekspresi dengan ``@(variabel) \ldots{}''.

\textgreater{}``@(a,b) a\textsuperscript{2+b}2'', \%(4,5)

\begin{verbatim}
@(a,b) a^2+b^2
41
\end{verbatim}

Ini memungkinkan untuk memanipulasi ekspresi dalam variabel lain untuk fungsi EMT yang membutuhkan ekspresi dalam ``x''.

Cara paling dasar untuk mendefinisikan fungsi sederhana adalah dengan menyimpan rumusnya dalam ekspresi simbolik atau numerik. Jika variabel utamanya adalah x, ekspresi dapat dievaluasi seperti fungsi.

Seperti yang Anda lihat dalam contoh berikut, variabel global terlihat selama evaluasi.

\textgreater fx \&= x\^{}3-a*x; \ldots{}\\
\textgreater{} a=1.2; fx(0.5)

\begin{verbatim}
-0.475
\end{verbatim}

Semua variabel lain dalam ekspresi dapat ditentukan dalam evaluasi menggunakan parameter yang ditetapkan.

\textgreater fx(0.5,a=1.1)

\begin{verbatim}
-0.425
\end{verbatim}

Sebuah ekspresi tidak harus simbolis. Ini perlu, jika ekspresi berisi fungsi, yang hanya diketahui dalam kernel numerik, bukan di Maxima.

\chapter{Matematika Simbolik}\label{matematika-simbolik}

EMT melakukan matematika simbolis dengan bantuan Maxima. Untuk detailnya, mulailah dengan tutorial berikut, atau telusuri referensi untuk Maxima. Para ahli di Maxima harus mencatat bahwa ada perbedaan sintaks antara sintaks asli Maxima dan sintaks default ekspresi simbolis di EMT.

Matematika simbolik diintegrasikan secara mulus ke dalam Euler dengan \&. Setiap ekspresi yang dimulai dengan \& adalah ekspresi simbolis. Ini dievaluasi dan dicetak oleh Maxima.

Pertama-tama, Maxima memiliki aritmatika ``tak terbatas'' yang dapat menangani angka yang sangat besar.

\textgreater\$\&44!

\[2658271574788448768043625811014615890319638528000000000\]Dengan cara ini, Anda dapat menghitung hasil yang besar dengan tepat. Mari kita hitung

lateks: C(44,10) = frac\{44!\} \{34! cdot 10!\}

\textgreater\$\& 44!/(34!*10!) // nilai C(44,10)

\[2481256778\]Tentu saja, Maxima memiliki fungsi yang lebih efisien untuk ini (seperti halnya bagian numerik EMT).

\textgreater\$binomial(44,10) //menghitung C(44,10) menggunakan fungsi binomial()

\[2481256778\]Untuk mempelajari lebih lanjut tentang fungsi tertentu, klik dua kali di atasnya. Misalnya, coba klik dua kali pada ``\&binomial'' di baris perintah sebelumnya. Ini membuka dokumentasi Maxima seperti yang disediakan oleh penulis program itu.

Anda akan belajar bahwa berikut ini juga berhasil.

lateks: C(x,3)=frac\{x!\} \{(x-3)!3!\} =frac\{(x-2)(x-1)x\}\{6\}

\textgreater\$binomial(x,3) // C(x,3)

\[\frac{\left(x-2\right)\,\left(x-1\right)\,x}{6}\]Jika Anda ingin mengganti x dengan nilai tertentu, gunakan ``with''.

\textgreater\$\&binomial(x,3) with x=10 // substitusi x=10 ke C(x,3)

\[120\]Dengan begitu Anda dapat menggunakan solusi dari suatu persamaan dalam persamaan lain.

Ekspresi simbolis dicetak oleh Maxima dalam bentuk 2D. Alasannya adalah bendera simbolis khusus dalam string.

Seperti yang akan Anda lihat di contoh sebelumnya dan berikutnya, jika Anda telah menginstal LaTeX, Anda dapat mencetak ekspresi simbolis dengan Latex. Jika tidak, perintah berikut akan mengeluarkan pesan kesalahan.

Untuk mencetak ekspresi simbolis dengan LaTeX, gunakan \$ di depan \& (atau Anda dapat menghilangkan \&) sebelum perintah. Jangan jalankan perintah Maxima dengan \$, jika Anda tidak menginstal LaTeX.

\textgreater\$(3+x)/(x\^{}2+1)

\[\frac{x+3}{x^2+1}\]Ekspresi simbolik diurai oleh Euler. Jika Anda memerlukan sintaks kompleks dalam satu ekspresi, Anda dapat menyertakan ekspresi dalam ``\ldots{}''. Untuk menggunakan lebih dari sekadar ungkapan sederhana dimungkinkan, tetapi sangat tidak dianjurkan.

\textgreater\&``v := 5; v\^{}2''

\begin{verbatim}
                                  25
\end{verbatim}

Untuk kelengkapan, kami mengatakan bahwa ekspresi simbolis dapat digunakan dalam program, tetapi perlu diapit dalam tanda kutip. Selain itu, jauh lebih efektif untuk memanggil Maxima pada waktu kompilasi jika memungkinkan.

\textgreater\$\&expand((1+x)\^{}4), \$\&factor(diff(\%,x)) // diff: turunan, factor: faktor

\[x^4+4\,x^3+6\,x^2+4\,x+1\] \[4\,\left(x+1\right)^3\]Sekali lagi, \% mengacu pada hasil sebelumnya.

Untuk mempermudah, kami menyimpan solusi ke variabel simbolis. Variabel simbolik didefinisikan dengan ``\&=''.

\textgreater fx \&= (x+1)/(x\^{}4+1); \$\&fx

\[\frac{x+1}{x^4+1}\]Ekspresi simbolis dapat digunakan dalam ekspresi simbolis lainnya.

\textgreater\$\&factor(diff(fx,x))

\[\frac{-3\,x^4-4\,x^3+1}{\left(x^4+1\right)^2}\]Input langsung dari perintah Maxima juga tersedia. Mulai baris perintah dengan ``::''. Sintaks Maxima disesuaikan dengan sintaks EMT (disebut ``mode kompatibilitas'').

\textgreater\&factor(20!)

\begin{verbatim}
                         2432902008176640000
\end{verbatim}

\textgreater::: factor(10!)

\begin{verbatim}
                               8  4  2
                              2  3  5  7
\end{verbatim}

\textgreater:: factor(20!)

\begin{verbatim}
                        18  8  4  2
                       2   3  5  7  11 13 17 19
\end{verbatim}

Jika Anda seorang ahli dalam Maxima, Anda mungkin ingin menggunakan sintaks asli Maxima. Anda dapat melakukan ini dengan ``:::''.

\textgreater::: av:g\$ av\^{}2;

\begin{verbatim}
                                   2
                                  g
\end{verbatim}

\textgreater fx \&= x\^{}3*exp(x), \$fx

\begin{verbatim}
                                 3  x
                                x  E
\end{verbatim}

\[x^3\,e^{x}\]Variabel tersebut dapat digunakan dalam ekspresi simbolis lainnya. Perhatikan bahwa dalam perintah berikut, sisi kanan \&= dievaluasi sebelum penetapan ke Fx.

\textgreater\&(fx with x=5), \$\%, \&float(\%)

\begin{verbatim}
                                     5
                                125 E
\end{verbatim}

\[125\,e^5\]\\
18551.64488782208

\textgreater fx(5)

\begin{verbatim}
18551.6448878
\end{verbatim}

Untuk evaluasi ekspresi dengan nilai variabel tertentu, Anda dapat menggunakan operator ``dengan''.

Baris perintah berikut juga menunjukkan bahwa Maxima dapat mengevaluasi ekspresi secara numerik dengan float().

\textgreater\&(fx with x=10)-(fx with x=5), \&float(\%)

\begin{verbatim}
                                10        5
                          1000 E   - 125 E


                         2.20079141499189e+7
\end{verbatim}

\textgreater\$factor(diff(fx,x,2))

\[x\,\left(x^2+6\,x+6\right)\,e^{x}\]Untuk mendapatkan kode Latex untuk ekspresi, Anda dapat menggunakan perintah tex.

\textgreater tex(fx)

\begin{verbatim}
x^3\,e^{x}
\end{verbatim}

Ekspresi simbolis dapat dievaluasi seperti ekspresi numerik.

\textgreater fx(0.5)

\begin{verbatim}
0.206090158838
\end{verbatim}

Dalam ekspresi simbolis, ini tidak berhasil, karena Maxima tidak mendukungnya. Sebagai gantinya, gunakan sintaks ``dengan'' (bentuk yang lebih baik dari perintah at(\ldots) Maxima).

\textgreater\$\&fx with x=1/2

\[\frac{\sqrt{e}}{8}\]Tugas juga bisa bersifat simbolis.

\textgreater\$\&fx with x=1+t

\[\left(t+1\right)^3\,e^{t+1}\]Perintah solve memecahkan ekspresi simbolis untuk variabel di Maxima. Hasilnya adalah vektor solusi.

\textgreater\$\&solve(x\^{}2+x=4,x)

\[\left[ x=\frac{-\sqrt{17}-1}{2} , x=\frac{\sqrt{17}-1}{2} \right]\]Bandingkan dengan perintah ``solve'' numerik di Euler, yang membutuhkan nilai awal, dan secara opsional nilai target.

\textgreater solve(``x\^{}2+x'',1,y=4)

\begin{verbatim}
1.56155281281
\end{verbatim}

Nilai numerik dari solusi simbolik dapat dihitung dengan evaluasi hasil simbolik. Euler akan membaca tugas x= dll. Jika Anda tidak memerlukan hasil numerik untuk perhitungan lebih lanjut, Anda juga dapat membiarkan Maxima menemukan nilai numerik.

\textgreater sol \&= solve(x\^{}2+2*x=4,x); \$\&sol, sol(), \$\&float(sol)

\[\left[ x=-\sqrt{5}-1 , x=\sqrt{5}-1 \right]\] {[}-3.23607, 1.23607{]} \[\left[ x=-3.23606797749979 , x=1.23606797749979 \right]\]Untuk mendapatkan solusi simbolis tertentu, seseorang dapat menggunakan ``dengan'' dan indeks.

\textgreater\$\&solve(x\^{}2+x=1,x), x2 \&= x with \%{[}2{]}; \$\&x2

\[\left[ x=\frac{-\sqrt{5}-1}{2} , x=\frac{\sqrt{5}-1}{2} \right]\] \[\frac{\sqrt{5}-1}{2}\]Untuk menyelesaikan sistem persamaan, gunakan vektor persamaan. Hasilnya adalah vektor solusi.

\textgreater sol \&= solve({[}x+y=3,x\textsuperscript{2+y}2=5{]},{[}x,y{]}); \$\&sol, \$\&x*y with sol{[}1{]}

\[\left[ \left[ x=2 , y=1 \right]  , \left[ x=1 , y=2 \right] \right]\] \[2\]Ekspresi simbolis dapat memiliki bendera, yang menunjukkan perlakuan khusus dalam Maxima. Beberapa bendera juga dapat digunakan sebagai perintah, yang lain tidak bisa. Bendera ditambahkan dengan ``\textbar{}'' (bentuk yang lebih baik dari ``ev(\ldots,flags)'')

\textgreater\$\& diff((x\^{}3-1)/(x+1),x) //turunan bentuk pecahan

\[\frac{3\,x^2}{x+1}-\frac{x^3-1}{\left(x+1\right)^2}\]\textgreater\$\& diff((x\^{}3-1)/(x+1),x) \textbar{} ratsimp //menyederhanakan pecahan

\[\frac{2\,x^3+3\,x^2+1}{x^2+2\,x+1}\]\textgreater\$\&factor(\%)

\[\frac{2\,x^3+3\,x^2+1}{\left(x+1\right)^2}\]*Fungsi

Dalam EMT, fungsi adalah program yang ditentukan dengan perintah ``fungsi''. Ini bisa berupa fungsi satu baris atau fungsi multibaris.

Fungsi satu baris dapat berupa numerik atau simbolis. Fungsi satu baris numerik didefinisikan oleh ``:=''.

\textgreater function f(x) := x*sqrt(x\^{}2+1)

Untuk gambaran umum, kami menunjukkan semua kemungkinan definisi untuk fungsi satu baris. Sebuah fungsi dapat dievaluasi seperti fungsi Euler bawaan.

\textgreater f(2)

\begin{verbatim}
4.472135955
\end{verbatim}

Fungsi ini juga akan berfungsi untuk vektor, mematuhi bahasa matriks Euler, karena ekspresi yang digunakan dalam fungsi ini divektorisasi.

\textgreater f(0:0.1:1)

\begin{verbatim}
[0,  0.100499,  0.203961,  0.313209,  0.430813,  0.559017,  0.699714,
0.854459,  1.0245,  1.21083,  1.41421]
\end{verbatim}

Fungsi dapat diplot. Alih-alih ekspresi, kita hanya perlu memberikan nama fungsi.

Berbeda dengan ekspresi simbolis atau numerik, nama fungsi harus disediakan dalam string.

\textgreater solve(``f'',1,y=1)

\begin{verbatim}
0.786151377757
\end{verbatim}

Secara default, jika Anda perlu menimpa fungsi bawaan, Anda harus menambahkan kata kunci ``menimpa''. Menimpa fungsi bawaan berbahaya dan dapat menyebabkan masalah untuk fungsi lain tergantung pada fungsi tersebut.

Anda masih dapat memanggil fungsi bawaan sebagai ``\_\ldots``, jika itu adalah fungsi di inti Euler.

\textgreater function overwrite sin (x) := \_sin(x°) // redine sine in degrees

\textgreater sin(45)

\begin{verbatim}
0.707106781187
\end{verbatim}

Lebih baik kita menghapus redefinisi sin.

\textgreater forget sin; sin(pi/4)

\begin{verbatim}
0.707106781187
\end{verbatim}

\section{Parameter Default}\label{parameter-default}

Fungsi numerik dapat memiliki parameter default.

\textgreater function f(x,a=1) := a*x\^{}2

Menghilangkan parameter ini menggunakan nilai default.

\textgreater f(4)

\begin{verbatim}
16
\end{verbatim}

Mengaturnya menimpa nilai default.

\textgreater f(4,5)

\begin{verbatim}
80
\end{verbatim}

Parameter yang ditetapkan juga menimpanya. Ini digunakan oleh banyak fungsi Euler seperti plot2d, plot3d.

\textgreater f(4,a=1)

\begin{verbatim}
16
\end{verbatim}

Jika variabel bukan parameter, variabel tersebut harus global. Fungsi satu baris dapat melihat variabel global.

\textgreater function f(x) := a*x\^{}2

\textgreater a=6; f(2)

\begin{verbatim}
24
\end{verbatim}

Tetapi parameter yang ditetapkan menimpa nilai global.

Jika argumen tidak ada dalam daftar parameter yang telah ditentukan sebelumnya, argumen harus dideklarasikan dengan ``:=''!

\textgreater f(2,a:=5)

\begin{verbatim}
20
\end{verbatim}

Fungsi simbolis didefinisikan dengan ``\&=''. Mereka didefinisikan dalam Euler dan Maxima, dan bekerja di kedua dunia. Ekspresi penentu dijalankan melalui Maxima sebelum definisi.

\textgreater function g(x) \&= x\^{}3-x*exp(-x); \$\&g(x)

\[x^3-x\,e^ {- x }\]Fungsi simbolis dapat digunakan dalam ekspresi simbolis.

\textgreater\$\&diff(g(x),x), \$\&\% with x=4/3

\[x\,e^ {- x }-e^ {- x }+3\,x^2\] \[\frac{e^ {- \frac{4}{3} }}{3}+\frac{16}{3}\]Mereka juga dapat digunakan dalam ekspresi numerik. Tentu saja, ini hanya akan berfungsi jika EMT dapat menafsirkan semua yang ada di dalam fungsi.

\textgreater g(5+g(1))

\begin{verbatim}
178.635099908
\end{verbatim}

Mereka dapat digunakan untuk mendefinisikan fungsi atau ekspresi simbolis lainnya.

\textgreater function G(x) \&= factor(integrate(g(x),x)); \$\&G(c) // integrate: mengintegralkan

\[\frac{e^ {- c }\,\left(c^4\,e^{c}+4\,c+4\right)}{4}\]\textgreater solve(\&g(x),0.5)

\begin{verbatim}
0.703467422498
\end{verbatim}

Berikut ini juga berfungsi, karena Euler menggunakan ekspresi simbolik dalam fungsi g, jika tidak menemukan variabel simbolik g, dan jika ada fungsi simbolik g.

\textgreater solve(\&g,0.5)

\begin{verbatim}
0.703467422498
\end{verbatim}

\textgreater function P(x,n) \&= (2*x-1)\^{}n; \$\&P(x,n)

\[\left(2\,x-1\right)^{n}\]\textgreater function Q(x,n) \&= (x+2)\^{}n; \$\&Q(x,n)

\[\left(x+2\right)^{n}\]\textgreater\$\&P(x,4), \$\&expand(\%)

\[\left(2\,x-1\right)^4\] \[16\,x^4-32\,x^3+24\,x^2-8\,x+1\]\textgreater P(3,4)

\begin{verbatim}
625
\end{verbatim}

\textgreater\$\&P(x,4)+ Q(x,3), \$\&expand(\%)

\[\left(2\,x-1\right)^4+\left(x+2\right)^3\] \[16\,x^4-31\,x^3+30\,x^2+4\,x+9\]\textgreater\$\&P(x,4)-Q(x,3), \$\&expand(\%), \$\&factor(\%)

\[\left(2\,x-1\right)^4-\left(x+2\right)^3\] \[16\,x^4-33\,x^3+18\,x^2-20\,x-7\] \[16\,x^4-33\,x^3+18\,x^2-20\,x-7\]\textgreater\$\&P(x,4)*Q(x,3), \$\&expand(\%), \$\&factor(\%)

\[\left(x+2\right)^3\,\left(2\,x-1\right)^4\] \[16\,x^7+64\,x^6+24\,x^5-120\,x^4-15\,x^3+102\,x^2-52\,x+8\] \[\left(x+2\right)^3\,\left(2\,x-1\right)^4\]\textgreater\$\&P(x,4)/Q(x,1), \$\&expand(\%), \$\&factor(\%)

\[\frac{\left(2\,x-1\right)^4}{x+2}\] \[\frac{16\,x^4}{x+2}-\frac{32\,x^3}{x+2}+\frac{24\,x^2}{x+2}-\frac{8\,x}{x+2}+\frac{1}{x+2}\] \[\frac{\left(2\,x-1\right)^4}{x+2}\]\textgreater function f(x) \&= x\^{}3-x; \$\&f(x)

\[x^3-x\]Dengan \&= fungsi bersifat simbolis, dan dapat digunakan dalam ekspresi simbolis lainnya.

\textgreater\$\&integrate(f(x),x)

\[\frac{x^4}{4}-\frac{x^2}{2}\]Dengan := fungsinya numerik. Contoh yang baik adalah integral pasti seperti

lateks: f(x) = int\_1\^{}x t\^{}t , dt,

yang tidak dapat dievaluasi secara simbolis.

Jika kita mendefinisikan ulang fungsi dengan kata kunci ``peta'', itu dapat digunakan untuk vektor x. Secara internal, fungsi dipanggil untuk semua nilai x sekali, dan hasilnya disimpan dalam vektor.

\textgreater function map f(x) := integrate(``x\^{}x'',1,x)

\textgreater f(0:0.5:2)

\begin{verbatim}
[-0.783431,  -0.410816,  0,  0.676863,  2.05045]
\end{verbatim}

Fungsi dapat memiliki nilai default untuk parameter.

\textgreater function mylog (x,base=10) := ln(x)/ln(base);

Sekarang fungsi dapat dipanggil dengan atau tanpa parameter ``basis''.

\textgreater mylog(100), mylog(2\^{}6.7,2)

\begin{verbatim}
2
6.7
\end{verbatim}

Selain itu, dimungkinkan untuk menggunakan parameter yang ditetapkan.

\textgreater mylog(E\^{}2,base=E)

\begin{verbatim}
2
\end{verbatim}

Seringkali, kita ingin menggunakan fungsi untuk vektor di satu tempat, dan untuk elemen individu di tempat lain. Ini dimungkinkan dengan parameter vektor.

\textgreater function f({[}a,b{]}) \&= a\textsuperscript{2+b}2-a*b+b; \$\&f(a,b), \$\&f(x,y)

\[b^2-a\,b+b+a^2\] \[y^2-x\,y+y+x^2\]Fungsi simbolis seperti itu dapat digunakan untuk variabel simbolik.

Tetapi fungsi ini juga dapat digunakan untuk vektor numerik.

\textgreater v={[}3,4{]}; f(v)

\begin{verbatim}
17
\end{verbatim}

Terdapat juga beberapa fungsi simbolik murni, yang tidak dapat digunakan secara numerik.

\textgreater function lapl(expr,x,y) \&\&= diff(expr,x,2)+diff(expr,y,2)//turunan parsial kedua

\begin{verbatim}
                 diff(expr, y, 2) + diff(expr, x, 2)
\end{verbatim}

\textgreater\$\&realpart((x+I*y)\^{}4), \$\&lapl(\%,x,y)

\[y^4-6\,x^2\,y^2+x^4\] \[0\]Namun tentu saja, mereka dapat digunakan dalam ekspresi simbolik atau dalam definisi fungsi simbolik.

\textgreater function f(x,y) \&= factor(lapl((x+y\textsuperscript{2)}5,x,y)); \$\&f(x,y)

\[10\,\left(y^2+x\right)^3\,\left(9\,y^2+x+2\right)\]Untuk meringkas

\begin{itemize}
\item
  \&= mendefinisikan fungsi simbolis,
\item
  := mendefinisikan fungsi numerik,
\item
  \&\&= mendefinisikan fungsi simbolis murni.
\end{itemize}

\chapter{Memecahkan Ekspresi}\label{memecahkan-ekspresi}

Ekspresi dapat diselesaikan secara numerik dan simbolis.

Untuk menyelesaikan ekspresi sederhana dari satu variabel, kita dapat menggunakan fungsi solve(). Perlu nilai awal untuk memulai pencarian. Secara internal, solve() menggunakan metode sekan.

\textgreater solve(``x\^{}2-2'',1)

\begin{verbatim}
1.41421356237
\end{verbatim}

Ini juga berfungsi untuk ekspresi simbolis. Ambil fungsi berikut.

\textgreater\$\&solve(x\^{}2=2,x)

\[\left[ x=-\sqrt{2} , x=\sqrt{2} \right]\]\textgreater\$\&solve(x\^{}2-2,x)

\[\left[ x=-\sqrt{2} , x=\sqrt{2} \right]\]\textgreater\$\&solve(a*x\^{}2+b*x+c=0,x)

\[\left[ x=\frac{-\sqrt{b^2-4\,a\,c}-b}{2\,a} , x=\frac{\sqrt{b^2-4\, a\,c}-b}{2\,a} \right]\]\textgreater\$\&solve({[}a*x+b*y=c,d*x+e*y=f{]},{[}x,y{]})

\[\left[ \left[ x=-\frac{c\,e}{b\,\left(d-5\right)-a\,e} , y=\frac{c \,\left(d-5\right)}{b\,\left(d-5\right)-a\,e} \right]\right]\]\textgreater px \&= 4*x\textsuperscript{8+x}7-x\^{}4-x; \$\&px

\[4\,x^8+x^7-x^4-x\]Sekarang kita mencari titik, di mana polinomial adalah 2. Dalam solve(), nilai target default y=0 dapat diubah dengan variabel yang ditetapkan.

Kami menggunakan y=2 dan memeriksa dengan mengevaluasi polinomial pada hasil sebelumnya.

\textgreater solve(px,1,y=2), px(\%)

\begin{verbatim}
0.966715594851
2
\end{verbatim}

Menyelesaikan ekspresi simbolis dalam bentuk simbolis mengembalikan daftar solusi. Kami menggunakan pemecah simbolis solve() yang disediakan oleh Maxima.

\textgreater sol \&= solve(x\^{}2-x-1,x); \$\&sol

\[\left[ x=\frac{1-\sqrt{5}}{2} , x=\frac{\sqrt{5}+1}{2} \right] \]Cara termudah untuk mendapatkan nilai numerik adalah dengan mengevaluasi solusi secara numerik seperti ekspresi.

\textgreater longest sol()

\begin{verbatim}
    -0.6180339887498949       1.618033988749895 
\end{verbatim}

Untuk menggunakan solusi secara simbolis dalam ekspresi lain, cara termudah adalah ``dengan''.

\textgreater\$\&x\^{}2 with sol{[}1{]}, \$\&expand(x\^{}2-x-1 with sol{[}2{]})

\[\frac{\left(\sqrt{5}-1\right)^2}{4}\] \[0\]Memecahkan sistem persamaan secara simbolis dapat dilakukan dengan vektor persamaan dan pemecah simbolik solve(). Jawabannya adalah daftar daftar persamaan.

\textgreater\$\&solve({[}x+y=2,x\^{}3+2*y+x=4{]},{[}x,y{]})

\[\left[ \left[ x=-1 , y=3 \right]  , \left[ x=1 , y=1 \right]  , \left[ x=0 , y=2 \right]  \right]\]Fungsi f() dapat melihat variabel global. Tetapi seringkali kita ingin menggunakan parameter lokal.

lateks: a\textsuperscript{x-x}a = 0.1

dengan a=3.

\textgreater function f(x,a) := x\textsuperscript{a-a}x;

Salah satu cara untuk meneruskan parameter tambahan ke f() adalah dengan menggunakan daftar dengan nama fungsi dan parameter (cara lain adalah parameter titik koma).

\textgreater solve(\{\{``f'',3\}\},2,y=0.1)

\begin{verbatim}
2.54116291558
\end{verbatim}

Ini juga berfungsi dengan ekspresi. Tapi kemudian, elemen daftar bernama harus digunakan. (Lebih lanjut tentang daftar dalam tutorial tentang sintaks EMT).

\textgreater solve(\{\{``x\textsuperscript{a-a}x'',a=3\}\},2,y=0.1)

\begin{verbatim}
2.54116291558
\end{verbatim}

\chapter{Menyelesaikan Pertidaksamaan}\label{menyelesaikan-pertidaksamaan}

Untuk menyelesaikan pertidaksamaan, EMT tidak akan dapat melakukannya, melainkan dengan bantuan Maxima, artinya secara eksak (simbolik). Perintah Maxima yang digunakan adalah fourier\_elim(), yang harus dipanggil dengan perintah ``load(fourier\_elim)'' terlebih dahulu.

\textgreater\&load(fourier\_elim)

\begin{verbatim}
        C:/Program Files/Euler x64/maxima/share/maxima/5.35.1/share/f\
ourier_elim/fourier_elim.lisp
\end{verbatim}

\textgreater\$\&fourier\_elim({[}x\^{}2 - 1\textgreater0{]},{[}x{]}) // x\^{}2-1 \textgreater{} 0

\[\left[ 1<x \right] \lor \left[ x<-1 \right]\]\textgreater\$\&fourier\_elim({[}x\^{}2 - 1\textless0{]},{[}x{]}) // x\^{}2-1 \textless{} 0

\[\left[ -1<x , x<1 \right]\]\textgreater\$\&fourier\_elim({[}x\^{}2 - 1 \# 0{]},{[}x{]}) // x\^{}-1 \textless\textgreater{} 0

\[\left[ -1<x , x<1 \right] \lor \left[ 1<x \right] \lor \left[ x<-1\right]\]\textgreater\$\&fourier\_elim({[}x \# 6{]},{[}x{]})

\[\left[ x<6 \right] \lor \left[ 6<x \right]\]\textgreater\$\&fourier\_elim({[}x \textless{} 1, x \textgreater{} 1{]},{[}x{]}) // tidak memiliki penyelesaian

\[{\it emptyset}\]\textgreater\$\&fourier\_elim({[}minf \textless{} x, x \textless{} inf{]},{[}x{]}) // solusinya R

\[{\it universalset}\]\textgreater\$\&fourier\_elim({[}x\^{}3 - 1 \textgreater{} 0{]},{[}x{]})

\[\left[ 1<x , x^2+x+1>0 \right] \lor \left[ x<1 , -x^2-x-1>0\right]\]\textgreater\$\&fourier\_elim({[}cos(x) \textless{} 1/2{]},{[}x{]}) // ??? gagal

\[\left[ 1-2\,\cos x>0 \right]\]\textgreater\$\&fourier\_elim({[}y-x \textless{} 5, x - y \textless{} 7, 10 \textless{} y{]},{[}x,y{]}) // sistem pertidaksamaan

\[\left[ y-5<x , x<y+7 , 10<y \right]\]\textgreater\$\&fourier\_elim({[}y-x \textless{} 5, x - y \textless{} 7, 10 \textless{} y{]},{[}y,x{]})

\[\left[ {\it max}\left(10 , x-7\right)<y , y<x+5 , 5<x \right]\]\textgreater\$\&fourier\_elim((x + y \textless{} 5) and (x - y \textgreater8),{[}x,y{]})

\[\left[ y+8<x , x<5-y , y<-\frac{3}{2} \right]\]\textgreater\$\&fourier\_elim(((x + y \textless{} 5) and x \textless{} 1) or (x - y \textgreater8),{[}x,y{]})

\[\left[ y+8<x \right] \lor \left[ x<{\it min}\left(1 , 5-y\right)\right]\]\textgreater\&fourier\_elim({[}max(x,y) \textgreater{} 6, x \# 8, abs(y-1) \textgreater{} 12{]},{[}x,y{]})

\begin{verbatim}
        [6 &lt; x, x &lt; 8, y &lt; - 11] or [8 &lt; x, y &lt; - 11]
 or [x &lt; 8, 13 &lt; y] or [x = y, 13 &lt; y] or [8 &lt; x, x &lt; y, 13 &lt; y]
 or [y &lt; x, 13 &lt; y]
\end{verbatim}

\textgreater\$\&fourier\_elim({[}(x+6)/(x-9) \textless= 6{]},{[}x{]})

\[\left[ x=12 \right] \lor \left[ 12<x \right] \lor \left[ x<9\right]\]\# Bahasa Matriks

Dokumentasi inti EMT berisi diskusi terperinci tentang bahasa matriks Euler.

Vektor dan matriks dimasukkan dengan tanda kurung siku, elemen yang dipisahkan oleh koma, baris yang dipisahkan oleh titik koma.

\textgreater A={[}1,2;3,4{]}

\begin{verbatim}
            1             2 
            3             4 
\end{verbatim}

Produk matriks dilambangkan dengan titik.

\textgreater b={[}3;4{]}

\begin{verbatim}
            3 
            4 
\end{verbatim}

\textgreater b' // transpose b

\begin{verbatim}
[3,  4]
\end{verbatim}

\textgreater inv(A) //inverse A

\begin{verbatim}
           -2             1 
          1.5          -0.5 
\end{verbatim}

\textgreater A.b //perkalian matriks

\begin{verbatim}
           11 
           25 
\end{verbatim}

\textgreater A.inv(A)

\begin{verbatim}
            1             0 
            0             1 
\end{verbatim}

Poin utama dari bahasa matriks adalah bahwa semua fungsi dan operator bekerja elemen demi elemen.

\textgreater A.A

\begin{verbatim}
            7            10 
           15            22 
\end{verbatim}

\textgreater A\^{}2 //perpangkatan elemen2 A

\begin{verbatim}
            1             4 
            9            16 
\end{verbatim}

\textgreater A.A.A

\begin{verbatim}
           37            54 
           81           118 
\end{verbatim}

\textgreater power(A,3) //perpangkatan matriks

\begin{verbatim}
           37            54 
           81           118 
\end{verbatim}

\textgreater A/A //pembagian elemen-elemen matriks yang seletak

\begin{verbatim}
            1             1 
            1             1 
\end{verbatim}

\textgreater A/b //pembagian elemen2 A oleh elemen2 b kolom demi kolom (karena b vektor kolom)

\begin{verbatim}
     0.333333      0.666667 
         0.75             1 
\end{verbatim}

\textgreater A\textbackslash b // hasilkali invers A dan b, A\^{}(-1)b

\begin{verbatim}
           -2 
          2.5 
\end{verbatim}

\textgreater inv(A).b

\begin{verbatim}
           -2 
          2.5 
\end{verbatim}

\textgreater A\textbackslash A //A\^{}(-1)A

\begin{verbatim}
            1             0 
            0             1 
\end{verbatim}

\textgreater inv(A).A

\begin{verbatim}
            1             0 
            0             1 
\end{verbatim}

\textgreater A*A //perkalin elemen-elemen matriks seletak

\begin{verbatim}
            1             4 
            9            16 
\end{verbatim}

Ini bukan produk matriks, tetapi elemen perkalian demi elemen. Hal yang sama berlaku untuk vektor.

\textgreater b\^{}2 // perpangkatan elemen-elemen matriks/vektor

\begin{verbatim}
            9 
           16 
\end{verbatim}

Jika salah satu operan adalah vektor atau skalar, itu diperluas dengan cara alami.

\textgreater2*A

\begin{verbatim}
            2             4 
            6             8 
\end{verbatim}

Misalnya, jika operan adalah vektor kolom, elemennya diterapkan ke semua baris A.

\textgreater{[}1,2{]}*A

\begin{verbatim}
            1             4 
            3             8 
\end{verbatim}

Jika itu adalah vektor baris, itu diterapkan ke semua kolom A.

\textgreater A*{[}2,3{]}

\begin{verbatim}
            2             6 
            6            12 
\end{verbatim}

Orang dapat membayangkan perkalian ini seolah-olah vektor baris v telah diduplikasi untuk membentuk matriks dengan ukuran yang sama dengan A.

\textgreater dup({[}1,2{]},2) // dup: menduplikasi/menggandakan vektor {[}1,2{]} sebanyak 2 kali (baris)

\begin{verbatim}
            1             2 
            1             2 
\end{verbatim}

\textgreater A*dup({[}1,2{]},2)

\begin{verbatim}
            1             4 
            3             8 
\end{verbatim}

Ini juga berlaku untuk dua vektor di mana satu adalah vektor baris dan yang lainnya adalah vektor kolom. Kami menghitung i*j untuk i,j dari 1 hingga 5. Triknya adalah mengalikan 1:5 dengan transposenya. Bahasa matriks Euler secara otomatis menghasilkan tabel nilai.

\textgreater(1:5)*(1:5)' // hasilkali elemen-elemen vektor baris dan vektor kolom

\begin{verbatim}
            1             2             3             4             5 
            2             4             6             8            10 
            3             6             9            12            15 
            4             8            12            16            20 
            5            10            15            20            25 
\end{verbatim}

Sekali lagi, ingatlah bahwa ini bukan produk matriks!

\textgreater(1:5).(1:5)' // hasilkali vektor baris dan vektor kolom

\begin{verbatim}
55
\end{verbatim}

\textgreater sum((1:5)*(1:5)) // sama hasilnya

\begin{verbatim}
55
\end{verbatim}

Bahkan operator seperti \textless{} atau == bekerja dengan cara yang sama.

\textgreater(1:10)\textless6 // menguji elemen-elemen yang kurang dari 6

\begin{verbatim}
[1,  1,  1,  1,  1,  0,  0,  0,  0,  0]
\end{verbatim}

Misalnya, kita dapat menghitung jumlah elemen yang memenuhi kondisi tertentu dengan fungsi sum().

\textgreater sum((1:10)\textless6) // banyak elemen yang kurang dari 6

\begin{verbatim}
5
\end{verbatim}

Euler memiliki operator perbandingan, seperti ``=='', yang memeriksa kesetaraan.

Kita mendapatkan vektor 0 dan 1, di mana 1 adalah singkatan dari true.

\textgreater t=(1:10)\^{}2; t==25 //menguji elemen2 t yang sama dengan 25 (hanya ada 1)

\begin{verbatim}
[0,  0,  0,  0,  1,  0,  0,  0,  0,  0]
\end{verbatim}

Dari vektor seperti itu, ``bukan nol'' memilih elemen bukan nol.

Dalam hal ini, kita mendapatkan indeks semua elemen yang lebih besar dari 50.

\textgreater nonzeros(t\textgreater50) //indeks elemen2 t yang lebih besar daripada 50

\begin{verbatim}
[8,  9,  10]
\end{verbatim}

Tentu saja, kita dapat menggunakan vektor indeks ini untuk mendapatkan nilai yang sesuai dalam t.

\textgreater t{[}nonzeros(t\textgreater50){]} //elemen2 t yang lebih besar daripada 50

\begin{verbatim}
[64,  81,  100]
\end{verbatim}

Sebagai contoh, mari kita temukan semua kuadrat dari angka 1 hingga 1000, yaitu 5 modulo 11 dan 3 modulo 13.

\textgreater t=1:1000; nonzeros(mod(t\^{}2,11)==5 \&\& mod(t\^{}2,13)==3)

\begin{verbatim}
[4,  48,  95,  139,  147,  191,  238,  282,  290,  334,  381,  425,
433,  477,  524,  568,  576,  620,  667,  711,  719,  763,  810,  854,
862,  906,  953,  997]
\end{verbatim}

EMT tidak sepenuhnya efektif untuk perhitungan bilangan bulat. Ini menggunakan titik mengambang presisi ganda secara internal. Namun, seringkali sangat berguna.

Kita dapat memeriksa keutamaan. Mari kita cari tahu, berapa banyak kotak ditambah 1 yang merupakan bilangan prima.

\textgreater t=1:1000; length(nonzeros(isprime(t\^{}2+1)))

\begin{verbatim}
112
\end{verbatim}

Fungsi nonzeros() hanya berfungsi untuk vektor. Untuk matriks, ada mnonzeros().

\textgreater seed(2); A=random(3,4)

\begin{verbatim}
     0.765761      0.401188      0.406347      0.267829 
      0.13673      0.390567      0.495975      0.952814 
     0.548138      0.006085      0.444255      0.539246 
\end{verbatim}

Ini mengembalikan indeks elemen, yang bukan nol.

\textgreater k=mnonzeros(A\textless0.4) //indeks elemen2 A yang kurang dari 0,4

\begin{verbatim}
            1             4 
            2             1 
            2             2 
            3             2 
\end{verbatim}

Indeks ini dapat digunakan untuk mengatur elemen ke beberapa nilai.

\textgreater mset(A,k,0) //mengganti elemen2 suatu matriks pada indeks tertentu

\begin{verbatim}
     0.765761      0.401188      0.406347             0 
            0             0      0.495975      0.952814 
     0.548138             0      0.444255      0.539246 
\end{verbatim}

Fungsi mset() juga dapat mengatur elemen pada indeks ke entri beberapa matriks lain.

\textgreater mset(A,k,-random(size(A)))

\begin{verbatim}
     0.765761      0.401188      0.406347     -0.126917 
    -0.122404     -0.691673      0.495975      0.952814 
     0.548138     -0.483902      0.444255      0.539246 
\end{verbatim}

Dan dimungkinkan untuk mendapatkan elemen dalam vektor.

\textgreater mget(A,k)

\begin{verbatim}
[0.267829,  0.13673,  0.390567,  0.006085]
\end{verbatim}

Fungsi lain yang berguna adalah extrema, yang mengembalikan nilai minimal dan maksimal di setiap baris matriks dan posisinya.

\textgreater ex=extrema(A)

\begin{verbatim}
     0.267829             4      0.765761             1 
      0.13673             1      0.952814             4 
     0.006085             2      0.548138             1 
\end{verbatim}

Kita dapat menggunakan ini untuk mengekstrak nilai maksimal di setiap baris.

\textgreater ex{[},3{]}'

\begin{verbatim}
[0.765761,  0.952814,  0.548138]
\end{verbatim}

Ini, tentu saja, sama dengan fungsi max().

\textgreater max(A)'

\begin{verbatim}
[0.765761,  0.952814,  0.548138]
\end{verbatim}

Tetapi dengan mget(), kita dapat mengekstrak indeks dan menggunakan informasi ini untuk mengekstrak elemen pada posisi yang sama dari matriks lain.

\textgreater j=(1:rows(A))'\textbar ex{[},4{]}, mget(-A,j)

\begin{verbatim}
            1             1 
            2             4 
            3             1 
[-0.765761,  -0.952814,  -0.548138]
\end{verbatim}

\chapter{Fungsi Matriks Lainnya (Matriks Bangunan)}\label{fungsi-matriks-lainnya-matriks-bangunan}

Untuk membangun matriks, kita dapat menumpuk satu matriks di atas matriks lainnya. Jika keduanya tidak memiliki jumlah kolom yang sama, yang lebih pendek akan diisi dengan 0.

\textgreater v=1:3; v\_v

\begin{verbatim}
            1             2             3 
            1             2             3 
\end{verbatim}

Demikian juga, kita dapat melampirkan matriks ke yang lain secara berdampingan, jika keduanya memiliki jumlah baris yang sama.

\textgreater A=random(3,4); A\textbar v'

\begin{verbatim}
     0.032444     0.0534171      0.595713      0.564454             1 
      0.83916      0.175552      0.396988       0.83514             2 
    0.0257573      0.658585      0.629832      0.770895             3 
\end{verbatim}

Jika mereka tidak memiliki jumlah baris yang sama, matriks yang lebih pendek diisi dengan 0.

Ada pengecualian untuk aturan ini. Bilangan real yang melekat pada matriks akan digunakan sebagai kolom yang diisi dengan bilangan real tersebut.

\textgreater A\textbar1

\begin{verbatim}
     0.032444     0.0534171      0.595713      0.564454             1 
      0.83916      0.175552      0.396988       0.83514             1 
    0.0257573      0.658585      0.629832      0.770895             1 
\end{verbatim}

Dimungkinkan untuk membuat matriks vektor baris dan kolom.

\textgreater{[}v;v{]}

\begin{verbatim}
            1             2             3 
            1             2             3 
\end{verbatim}

\textgreater{[}v',v'{]}

\begin{verbatim}
            1             1 
            2             2 
            3             3 
\end{verbatim}

Tujuan utamanya adalah untuk menafsirkan vektor ekspresi untuk vektor kolom.

\textgreater{}``{[}x,x\^{}2{]}''(v')

\begin{verbatim}
            1             1 
            2             4 
            3             9 
\end{verbatim}

Untuk mendapatkan ukuran A, kita bisa menggunakan fungsi berikut.

\textgreater C=zeros(2,4); rows(C), cols(C), size(C), length(C)

\begin{verbatim}
2
4
[2,  4]
4
\end{verbatim}

Untuk vektor, ada length().

\textgreater length(2:10)

\begin{verbatim}
9
\end{verbatim}

Ada banyak fungsi lain, yang menghasilkan matriks.

\textgreater ones(2,2)

\begin{verbatim}
            1             1 
            1             1 
\end{verbatim}

Ini juga dapat digunakan dengan satu parameter. Untuk mendapatkan vektor dengan angka lain selain 1, gunakan yang berikut ini.

\textgreater ones(5)*6

\begin{verbatim}
[6,  6,  6,  6,  6]
\end{verbatim}

Juga matriks bilangan acak dapat dihasilkan dengan acak (distribusi seragam) atau normal (distribusi Gauß).

\textgreater random(2,2)

\begin{verbatim}
      0.66566      0.831835 
        0.977      0.544258 
\end{verbatim}

Berikut adalah fungsi berguna lainnya, yang merestrukturisasi elemen matriks menjadi matriks lain.

\textgreater redim(1:9,3,3) // menyusun elemen2 1, 2, 3, \ldots, 9 ke bentuk matriks 3x3

\begin{verbatim}
            1             2             3 
            4             5             6 
            7             8             9 
\end{verbatim}

Dengan fungsi berikut, kita dapat menggunakan ini dan fungsi dup untuk menulis fungsi rep(), yang mengulangi vektor n kali.

\textgreater function rep(v,n) := redim(dup(v,n),1,n*cols(v))

Let us test.

\textgreater rep(1:3,5)

\begin{verbatim}
[1,  2,  3,  1,  2,  3,  1,  2,  3,  1,  2,  3,  1,  2,  3]
\end{verbatim}

Fungsi multdup() menduplikasi elemen vektor.

\textgreater multdup(1:3,5), multdup(1:3,{[}2,3,2{]})

\begin{verbatim}
[1,  1,  1,  1,  1,  2,  2,  2,  2,  2,  3,  3,  3,  3,  3]
[1,  1,  2,  2,  2,  3,  3]
\end{verbatim}

Fungsi flipx() dan flipy() mengembalikan urutan baris atau kolom matriks. Yaitu, fungsi flipx() membalik secara horizontal.

\textgreater flipx(1:5) //membalik elemen2 vektor baris

\begin{verbatim}
[5,  4,  3,  2,  1]
\end{verbatim}

Untuk rotasi, Euler memiliki rotleft() dan rotright().

\textgreater rotleft(1:5) // memutar elemen2 vektor baris

\begin{verbatim}
[2,  3,  4,  5,  1]
\end{verbatim}

Fungsi khusus adalah drop(v,i), yang menghapus elemen dengan indeks di i dari vektor v.

\textgreater drop(10:20,3)

\begin{verbatim}
[10,  11,  13,  14,  15,  16,  17,  18,  19,  20]
\end{verbatim}

Perhatikan bahwa vektor i dalam drop(v,i) mengacu pada indeks elemen dalam v, bukan nilai elemen. Jika Anda ingin menghapus elemen, Anda harus menemukan elemen terlebih dahulu. Fungsi indexof(v,x) dapat digunakan untuk menemukan elemen x dalam vektor yang diurutkan v.

\textgreater v=primes(50), i=indexof(v,10:20), drop(v,i)

\begin{verbatim}
[2,  3,  5,  7,  11,  13,  17,  19,  23,  29,  31,  37,  41,  43,  47]
[0,  5,  0,  6,  0,  0,  0,  7,  0,  8,  0]
[2,  3,  5,  7,  23,  29,  31,  37,  41,  43,  47]
\end{verbatim}

Seperti yang Anda lihat, tidak ada salahnya memasukkan indeks di luar jangkauan (seperti 0), indeks ganda, atau indeks yang tidak diurutkan.

\textgreater drop(1:10,shuffle({[}0,0,5,5,7,12,12{]}))

\begin{verbatim}
[1,  2,  3,  4,  6,  8,  9,  10]
\end{verbatim}

Ada beberapa fungsi khusus untuk mengatur diagonal atau untuk menghasilkan matriks diagonal.

Kita mulai dengan matriks identitas.

\textgreater A=id(5) // matriks identitas 5x5

\begin{verbatim}
            1             0             0             0             0 
            0             1             0             0             0 
            0             0             1             0             0 
            0             0             0             1             0 
            0             0             0             0             1 
\end{verbatim}

Kemudian kita mengatur diagonal bawah (-1) menjadi 1:4.

\textgreater setdiag(A,-1,1:4) //mengganti diagonal di bawah diagonal utama

\begin{verbatim}
            1             0             0             0             0 
            1             1             0             0             0 
            0             2             1             0             0 
            0             0             3             1             0 
            0             0             0             4             1 
\end{verbatim}

Perhatikan bahwa kami tidak mengubah matriks A. Kita mendapatkan matriks baru sebagai hasil dari setdiag().

Berikut adalah fungsi, yang mengembalikan matriks tri-diagonal.

\textgreater function tridiag (n,a,b,c) := setdiag(setdiag(b*id(n),1,c),-1,a); \ldots{}\\
\textgreater{} tridiag(5,1,2,3)

\begin{verbatim}
            2             3             0             0             0 
            1             2             3             0             0 
            0             1             2             3             0 
            0             0             1             2             3 
            0             0             0             1             2 
\end{verbatim}

Diagonal matriks juga dapat diekstraksi dari matriks. Untuk mendemonstrasikan ini, kami merestrukturisasi vektor 1:9 menjadi matriks 3x3.

\textgreater A=redim(1:9,3,3)

\begin{verbatim}
            1             2             3 
            4             5             6 
            7             8             9 
\end{verbatim}

Sekarang kita bisa mengekstrak diagonal.

\textgreater d=getdiag(A,0)

\begin{verbatim}
[1,  5,  9]
\end{verbatim}

Misalnya Kita dapat membagi matriks dengan diagonalnya. Bahasa matriks menjaga bahwa vektor kolom d diterapkan ke matriks baris demi baris.

\textgreater fraction A/d'

\begin{verbatim}
        1         2         3 
      4/5         1       6/5 
      7/9       8/9         1 
\end{verbatim}

\chapter{Vektorisasi}\label{vektorisasi}

Hampir semua fungsi di Euler bekerja untuk input matriks dan vektor juga, kapan pun ini masuk akal.

Misalnya, fungsi sqrt() menghitung akar kuadrat dari semua elemen vektor atau matriks.

\textgreater sqrt(1:3)

\begin{verbatim}
[1,  1.41421,  1.73205]
\end{verbatim}

Jadi Anda dapat dengan mudah membuat tabel nilai. Ini adalah salah satu cara untuk memplot fungsi (alternatif menggunakan ekspresi).

\textgreater x=1:0.01:5; y=log(x)/x\^{}2; // terlalu panjang untuk ditampikan

Dengan ini dan operator titik dua a:delta:b, vektor nilai fungsi dapat dihasilkan dengan mudah.

Dalam contoh berikut, kita menghasilkan vektor nilai t{[}i{]} dengan spasi 0,1 dari -1 hingga 1. Kemudian kita menghasilkan vektor nilai fungsi

lateks: s = t\^{}3-t

\textgreater t=-1:0.1:1; s=t\^{}3-t

\begin{verbatim}
[0,  0.171,  0.288,  0.357,  0.384,  0.375,  0.336,  0.273,  0.192,
0.099,  0,  -0.099,  -0.192,  -0.273,  -0.336,  -0.375,  -0.384,
-0.357,  -0.288,  -0.171,  0]
\end{verbatim}

EMT memperluas operator untuk skalar, vektor, dan matriks dengan cara yang jelas.

Misalnya, vektor kolom dikalikan vektor baris meluas ke matriks, jika operator diterapkan. Berikut ini, v' adalah vektor yang ditransposkan (vektor kolom).

\textgreater shortest (1:5)*(1:5)'

\begin{verbatim}
     1      2      3      4      5 
     2      4      6      8     10 
     3      6      9     12     15 
     4      8     12     16     20 
     5     10     15     20     25 
\end{verbatim}

Catatan, bahwa ini sangat berbeda dengan produk matriks. Produk matriks dilambangkan dengan titik ``.'' di EMT.

\textgreater(1:5).(1:5)'

\begin{verbatim}
55
\end{verbatim}

Secara default, vektor baris dicetak dalam format yang ringkas.

\textgreater{[}1,2,3,4{]}

\begin{verbatim}
[1,  2,  3,  4]
\end{verbatim}

Untuk matriks operator khusus . menunjukkan perkalian matriks, dan A' menunjukkan transposisi. Matriks 1x1 dapat digunakan seperti bilangan real.

\textgreater v:={[}1,2{]}; v.v', \%\^{}2

\begin{verbatim}
5
25
\end{verbatim}

Untuk mentransposisi matriks, kita menggunakan apostrof.

\textgreater v=1:4; v'

\begin{verbatim}
            1 
            2 
            3 
            4 
\end{verbatim}

Jadi kita dapat menghitung matriks A kali vektor b.

\textgreater A={[}1,2,3,4;5,6,7,8{]}; A.v'

\begin{verbatim}
           30 
           70 
\end{verbatim}

Perhatikan bahwa v masih merupakan vektor baris. Jadi v'.v berbeda dari v.v'.

\textgreater v'.v

\begin{verbatim}
            1             2             3             4 
            2             4             6             8 
            3             6             9            12 
            4             8            12            16 
\end{verbatim}

v.v' menghitung norma v kuadrat untuk vektor baris v. Hasilnya adalah vektor 1x1, yang bekerja seperti bilangan real.

\textgreater v.v'

\begin{verbatim}
30
\end{verbatim}

Ada juga norma fungsi (bersama dengan banyak fungsi lain dari Aljabar Linier).

\textgreater norm(v)\^{}2

\begin{verbatim}
30
\end{verbatim}

Operator dan fungsi mematuhi bahasa matriks Euler.

Berikut adalah ringkasan aturannya.

\begin{itemize}
\item
  Fungsi yang diterapkan pada vektor atau matriks diterapkan pada
\item
  setiap elemen.
\item
  Operator yang beroperasi pada dua matriks dengan ukuran yang sama
\item
  diterapkan secara berpasangan ke elemen matriks.
\item
  Jika kedua matriks memiliki dimensi yang berbeda, keduanya diperluas
\item
  dengan cara yang masuk akal, sehingga memiliki ukuran yang sama.
\end{itemize}

Misalnya, nilai skalar dikalikan vektor mengalikan nilai dengan setiap elemen vektor. Atau matriks dikalikan vektor (dengan *, bukan .) memperluas vektor ke ukuran matriks dengan menduplikasinya.

Berikut ini adalah kasus sederhana dengan operator \^{}.

\textgreater{[}1,2,3{]}\^{}2

\begin{verbatim}
[1,  4,  9]
\end{verbatim}

Ini adalah kasus yang lebih rumit. Vektor baris dikalikan vektor kolom memperluas keduanya dengan menduplikasi.

\textgreater v:={[}1,2,3{]}; v*v'

\begin{verbatim}
            1             2             3 
            2             4             6 
            3             6             9 
\end{verbatim}

Perhatikan bahwa produk skalar menggunakan produk matriks, bukan *!

\textgreater v.v'

\begin{verbatim}
14
\end{verbatim}

Ada banyak fungsi untuk matriks. Kami memberikan daftar pendek. Anda harus berkonsultasi dengan dokumentasi untuk informasi lebih lanjut tentang perintah ini.

sum,prod menghitung jumlah dan produk dari baris

cumsum, cumprod melakukan hal yang sama secara kumulatif

menghitung nilai ekstrem setiap baris

extrema mengembalikan vektor dengan informasi ekstrem

diag(A,i) mengembalikan diagonal ke-i

setdiag(A,i,v) mengatur diagonal ke-i

id(n) matriks identitas

det(A) penentu

charpoly(A) polinomial karakteristik

nilai eigen(A) nilai eigen

\textgreater v*v, sum(v*v), cumsum(v*v)

\begin{verbatim}
[1,  4,  9]
14
[1,  5,  14]
\end{verbatim}

Operator : menghasilkan vektor baris spasi yang sama, opsional dengan ukuran langkah.

\textgreater1:4, 1:2:10

\begin{verbatim}
[1,  2,  3,  4]
[1,  3,  5,  7,  9]
\end{verbatim}

Untuk menggabungkan matriks dan vektor ada operator ``\textbar{}'' dan ``\_``.

\textgreater{[}1,2,3{]}\textbar{[}4,5{]}, {[}1,2,3{]}\_1

\begin{verbatim}
[1,  2,  3,  4,  5]
            1             2             3 
            1             1             1 
\end{verbatim}

Unsur-unsur matriks disebut dengan ``A{[}i,j{]}''.

\textgreater A:={[}1,2,3;4,5,6;7,8,9{]}; A{[}2,3{]}

\begin{verbatim}
6
\end{verbatim}

Untuk vektor baris atau kolom, v{[}i{]} adalah elemen ke-i dari vektor. Untuk matriks, ini mengembalikan baris ke-i lengkap dari matriks.

\textgreater v:={[}2,4,6,8{]}; v{[}3{]}, A{[}3{]}

\begin{verbatim}
6
[7,  8,  9]
\end{verbatim}

Indeks juga dapat berupa vektor baris indeks. : menunjukkan semua indeks.

\textgreater v{[}1:2{]}, A{[}:,2{]}

\begin{verbatim}
[2,  4]
            2 
            5 
            8 
\end{verbatim}

Bentuk singkat untuk : menghilangkan indeks sepenuhnya.

\textgreater A{[},2:3{]}

\begin{verbatim}
            2             3 
            5             6 
            8             9 
\end{verbatim}

Untuk tujuan vektorisasi, elemen matriks dapat diakses seolah-olah mereka adalah vektor.

\textgreater A\{4\}

\begin{verbatim}
4
\end{verbatim}

Matriks juga dapat diratakan, menggunakan fungsi redim(). Ini diimplementasikan dalam fungsi flatten().

\textgreater redim(A,1,prod(size(A))), flatten(A)

\begin{verbatim}
[1,  2,  3,  4,  5,  6,  7,  8,  9]
[1,  2,  3,  4,  5,  6,  7,  8,  9]
\end{verbatim}

Untuk menggunakan matriks untuk tabel, mari kita atur ulang ke format default, dan menghitung tabel nilai sinus dan kosinus. Perhatikan bahwa sudut dalam radian secara default.

\textgreater defformat; w=0°:45°:360°; w=w'; deg(w)

\begin{verbatim}
            0 
           45 
           90 
          135 
          180 
          225 
          270 
          315 
          360 
\end{verbatim}

Sekarang kita menambahkan kolom ke matriks.

\textgreater M = deg(w)\textbar w\textbar cos(w)\textbar sin(w)

\begin{verbatim}
            0             0             1             0 
           45      0.785398      0.707107      0.707107 
           90        1.5708             0             1 
          135       2.35619     -0.707107      0.707107 
          180       3.14159            -1             0 
          225       3.92699     -0.707107     -0.707107 
          270       4.71239             0            -1 
          315       5.49779      0.707107     -0.707107 
          360       6.28319             1             0 
\end{verbatim}

Dengan menggunakan bahasa matriks, kita dapat menghasilkan beberapa tabel dari beberapa fungsi sekaligus.

Dalam contoh berikut, kita menghitung t{[}j{]}\^{}i untuk i dari 1 hingga n. Kita mendapatkan matriks, di mana setiap baris adalah tabel t\^{}i untuk satu i. Yaitu, matriks memiliki unsur lateks: a\_\{i,j\} = t\_j\^{}i, quad 1 le j le 101, quad 1 le i le n

Fungsi yang tidak berfungsi untuk input vektor harus ``vektorisasi''. Ini dapat dicapai dengan kata kunci ``peta'' dalam definisi fungsi. Kemudian fungsi akan dievaluasi untuk setiap elemen parameter vektor.

Integrasi numerik integrate() hanya berfungsi untuk batas interval skalar. Jadi kita perlu membuat vektorisasi.

\textgreater function map f(x) := integrate(``x\^{}x'',1,x)

Kata kunci ``peta'' menvektor fungsi. Fungsi sekarang akan berfungsi

untuk vektor angka.

\textgreater f({[}1:5{]})

\begin{verbatim}
[0,  2.05045,  13.7251,  113.336,  1241.03]
\end{verbatim}

\chapter{Sub-Matriks dan Elemen Matriks}\label{sub-matriks-dan-elemen-matriks}

Untuk mengakses elemen matriks, gunakan notasi tanda kurung.

\textgreater A={[}1,2,3;4,5,6;7,8,9{]}, A{[}2,2{]}

\begin{verbatim}
            1             2             3 
            4             5             6 
            7             8             9 
5
\end{verbatim}

Kita dapat mengakses baris lengkap matriks.

\textgreater A{[}2{]}

\begin{verbatim}
[4,  5,  6]
\end{verbatim}

Dalam kasus vektor baris atau kolom, ini mengembalikan elemen vektor.

\textgreater v=1:3; v{[}2{]}

\begin{verbatim}
2
\end{verbatim}

Untuk memastikan, Anda mendapatkan baris pertama untuk matriks 1xn dan mxn, tentukan semua kolom menggunakan indeks kedua yang kosong.

\textgreater A{[}2,{]}

\begin{verbatim}
[4,  5,  6]
\end{verbatim}

Jika indeks adalah vektor indeks, Euler akan mengembalikan baris matriks yang sesuai.

Di sini kita menginginkan baris pertama dan kedua A.

\begin{verbatim}
            1             2             3 
            4             5             6 
\end{verbatim}

WKita bahkan dapat menyusun ulang A menggunakan vektor indeks. Tepatnya, kami tidak mengubah A di sini, tetapi menghitung versi A yang disusun ulang.

\begin{verbatim}
            7             8             9 
            4             5             6 
            1             2             3 
\end{verbatim}

Trik indeks juga bekerja dengan kolom.

Contoh ini memilih semua baris A dan kolom kedua dan ketiga.

\textgreater A{[}1:3,2:3{]}

\begin{verbatim}
            2             3 
            5             6 
            8             9 
\end{verbatim}

Untuk singkatan ``:'' menunjukkan semua indeks baris atau kolom.

\textgreater A{[}:,3{]}

\begin{verbatim}
            3 
            6 
            9 
\end{verbatim}

Atau, biarkan indeks pertama kosong.

\textgreater A{[},2:3{]}

\begin{verbatim}
            2             3 
            5             6 
            8             9 
\end{verbatim}

Kita juga bisa mendapatkan baris terakhir A.

\textgreater A{[}-1{]}

\begin{verbatim}
[7,  8,  9]
\end{verbatim}

Sekarang mari kita ubah elemen A dengan menetapkan submatriks A ke beberapa nilai. Ini sebenarnya mengubah matriks A yang tersimpan.

\textgreater A{[}1,1{]}=4

\begin{verbatim}
            4             2             3 
            4             5             6 
            7             8             9 
\end{verbatim}

Kita juga dapat menetapkan nilai ke baris A.

\textgreater A{[}1{]}={[}-1,-1,-1{]}

\begin{verbatim}
           -1            -1            -1 
            4             5             6 
            7             8             9 
\end{verbatim}

Kita bahkan dapat menetapkan ke sub-matriks jika memiliki ukuran yang tepat.

\textgreater A{[}1:2,1:2{]}={[}5,6;7,8{]}

\begin{verbatim}
            5             6            -1 
            7             8             6 
            7             8             9 
\end{verbatim}

Selain itu, beberapa jalan pintas diperbolehkan.

\textgreater A{[}1:2,1:2{]}=0

\begin{verbatim}
            0             0            -1 
            0             0             6 
            7             8             9 
\end{verbatim}

Peringatan: Indeks di luar batas mengembalikan matriks kosong, atau pesan kesalahan, tergantung pada pengaturan sistem. Defaultnya adalah pesan kesalahan. Ingat, bagaimanapun, bahwa indeks negatif dapat digunakan untuk mengakses elemen matriks yang dihitung dari akhir.

\textgreater A{[}4{]}

\begin{verbatim}
Row index 4 out of bounds!
Error in:
A[4] ...
    ^
\end{verbatim}

\chapter{Penyortiran dan Pengocokan}\label{penyortiran-dan-pengocokan}

Fungsi sort() mengurutkan vektor baris.

\textgreater sort({[}5,6,4,8,1,9{]})

\begin{verbatim}
[1,  4,  5,  6,  8,  9]
\end{verbatim}

Seringkali perlu untuk mengetahui indeks vektor yang diurutkan dalam vektor asli. Ini dapat digunakan untuk menyusun ulang vektor lain dengan cara yang sama.

Mari kita acak vektor.

\textgreater v=shuffle(1:10)

\begin{verbatim}
[4,  5,  10,  6,  8,  9,  1,  7,  2,  3]
\end{verbatim}

Indeks berisi urutan yang tepat dari v.

\textgreater\{vs,ind\}=sort(v); v{[}ind{]}

\begin{verbatim}
[1,  2,  3,  4,  5,  6,  7,  8,  9,  10]
\end{verbatim}

Ini juga berfungsi untuk vektor string.

\textgreater s={[}``a'',``d'',``e'',``a'',``aa'',``e''{]}

\begin{verbatim}
a
d
e
a
aa
e
\end{verbatim}

\textgreater\{ss,ind\}=sort(s); ss

\begin{verbatim}
a
a
aa
d
e
e
\end{verbatim}

Seperti yang Anda lihat, posisi entri ganda agak acak.

\textgreater ind

\begin{verbatim}
[4,  1,  5,  2,  6,  3]
\end{verbatim}

Fungsi unique mengembalikan daftar elemen unik vektor.

\textgreater intrandom(1,10,10), unique(\%)

\begin{verbatim}
[4,  4,  9,  2,  6,  5,  10,  6,  5,  1]
[1,  2,  4,  5,  6,  9,  10]
\end{verbatim}

Ini juga berfungsi untuk vektor string.

\textgreater unique(s)

\begin{verbatim}
a
aa
d
e
\end{verbatim}

\chapter{Aljabar Linier}\label{aljabar-linier}

EMT memiliki banyak fungsi untuk memecahkan sistem linier, sistem jarang, atau masalah regresi.

Untuk sistem linier Ax=b, Anda dapat menggunakan algoritma Gauss, matriks terbalik atau kecocokan linier. Operator Ab menggunakan versi algoritma Gauss.

\textgreater A={[}1,2;3,4{]}; b={[}5;6{]}; A\textbackslash b

\begin{verbatim}
           -4 
          4.5 
\end{verbatim}

Untuk contoh lain, kita menghasilkan matriks 200x200 dan jumlah barisnya. Kemudian kita selesaikan Ax=b menggunakan matriks terbalik. Kami mengukur kesalahan sebagai penyimpangan maksimum dari semua elemen dari 1, yang tentu saja merupakan solusi yang benar.

\textgreater A=normal(200,200); b=sum(A); longest totalmax(abs(inv(A).b-1))

\begin{verbatim}
  8.790745908981989e-13 
\end{verbatim}

Jika sistem tidak memiliki solusi, kecocokan linier meminimalkan norma kesalahan Ax-b.

\textgreater A={[}1,2,3;4,5,6;7,8,9{]}

\begin{verbatim}
            1             2             3 
            4             5             6 
            7             8             9 
\end{verbatim}

Penentu matriks ini adalah 0.

\textgreater det(A)

\begin{verbatim}
0
\end{verbatim}

\chapter{Matriks Simbolis}\label{matriks-simbolis}

Maxima memiliki matriks simbolik. Tentu saja, Maxima dapat digunakan untuk masalah aljabar linier sederhana seperti itu. Kita dapat mendefinisikan matriks untuk Euler dan Maxima dengan \&:=, dan kemudian menggunakannya dalam ekspresi simbolik. Bentuk {[}\ldots{]} yang biasa untuk mendefinisikan matriks dapat digunakan dalam Euler untuk mendefinisikan matriks simbolik.

\textgreater A \&= {[}a,1,1;1,a,1;1,1,a{]}; \$A

\textgreater\$\&det(A), \$\&factor(\%)

\textgreater\$\&invert(A) with a=0

\textgreater A \&= {[}1,a;b,2{]}; \$A

Seperti semua variabel simbolik, matriks ini dapat digunakan dalam ekspresi simbolis lainnya.

\textgreater\$\&det(A-x*ident(2)), \$\&solve(\%,x)

Nilai eigen juga dapat dihitung secara otomatis. Hasilnya adalah vektor dengan dua vektor nilai eigen dan multiplisitas.

\textgreater\$\&eigenvalues({[}a,1;1,a{]})

Untuk mengekstrak eigenvector tertentu membutuhkan pengindeksan yang cermat.

\textgreater\$\&eigenvectors({[}a,1;1,a{]}), \&\%{[}2{]}{[}1{]}{[}1{]}

\begin{verbatim}
                               [1, - 1]
\end{verbatim}

Matriks simbolik dapat dievaluasi dalam Euler secara numerik seperti ekspresi simbolik lainnya.

\textgreater A(a=4,b=5)

\begin{verbatim}
            1             4 
            5             2 
\end{verbatim}

Dalam ekspresi simbolis, gunakan dengan.

\textgreater\$\&A with {[}a=4,b=5{]}

Akses ke baris matriks simbolis berfungsi seperti matriks numerik.

\textgreater\$\&A{[}1{]}

Ekspresi simbolis dapat berisi tugas. Dan itu mengubah matriks A.

\textgreater\&A{[}1,1{]}:=t+1; \$\&A

Ada fungsi simbolik di Maxima untuk membuat vektor dan matriks. Untuk ini, lihat dokumentasi Maxima atau tutorial tentang Maxima di EMT.

\textgreater v \&= makelist(1/(i+j),i,1,3); \$v

\textgreater B \&:= {[}1,2;3,4{]}; \$B, \$\&invert(B)

Hasilnya dapat dievaluasi secara numerik dalam Euler. Untuk informasi lebih lanjut tentang Maxima, lihat pengantar Maxima.

\textgreater\$\&invert(B)()

\begin{verbatim}
           -2             1 
          1.5          -0.5 
\end{verbatim}

Euler juga memiliki fungsi yang kuat xinv(), yang melakukan upaya yang lebih besar dan mendapatkan hasil yang lebih tepat.

Perhatikan, bahwa dengan \&:= matriks B telah didefinisikan sebagai simbolik dalam ekspresi simbolik dan sebagai numerik dalam ekspresi numerik. Jadi kita bisa menggunakannya di sini.

\textgreater longest B.xinv(B)

\begin{verbatim}
                      1                       0 
                      0                       1 
\end{verbatim}

Misalnya nilai eigen A dapat dihitung secara numerik.

\textgreater A={[}1,2,3;4,5,6;7,8,9{]}; real(eigenvalues(A))

\begin{verbatim}
[16.1168,  -1.11684,  0]
\end{verbatim}

Atau secara simbolis. Lihat tutorial tentang Maxima untuk detail tentang ini.

\textgreater\$\&eigenvalues((\textbf{A?}))

\chapter{Nilai Numerik dalam Ekspresi Simbolis}\label{nilai-numerik-dalam-ekspresi-simbolis}

Ekspresi simbolis hanyalah string yang berisi ekspresi. Jika kita ingin mendefinisikan nilai baik untuk ekspresi simbolik maupun untuk ekspresi numerik, kita harus menggunakan ``\&:=''.

\textgreater A \&:= {[}1,pi;4,5{]}

\begin{verbatim}
            1       3.14159 
            4             5 
\end{verbatim}

Masih ada perbedaan antara bentuk numerik dan simbolis. Saat mentransfer matriks ke bentuk simbolik, perkiraan pecahan untuk real akan digunakan.

\textgreater\$\&A

Untuk menghindari hal ini, ada fungsi ``mxmset(variabel)''.

\textgreater mxmset(A); \$\&A

Maxima juga dapat menghitung dengan angka floating point, dan bahkan dengan angka mengambang besar dengan 32 digit. Namun, evaluasinya jauh lebih lambat.

\textgreater\$\&bfloat(sqrt(2)), \$\&float(sqrt(2))

Ketepatan angka floating point besar dapat diubah.

\textgreater\&fpprec:=100; \&bfloat(pi)

\begin{verbatim}
        3.14159265358979323846264338327950288419716939937510582097494\
4592307816406286208998628034825342117068b0
\end{verbatim}

Variabel numerik dapat digunakan dalam ekspresi simbolis apa pun menggunakan ``(\textbf{var?})''.

Perhatikan bahwa ini hanya diperlukan, jika variabel telah didefinisikan dengan ``:='' atau ``='' sebagai variabel numerik.

\textgreater B:={[}1,pi;3,4{]}; \$\&det((\textbf{B?}))

\chapter{Demo - Suku Bunga}\label{demo---suku-bunga}

Di bawah ini, kami menggunakan Euler Math Toolbox (EMT) untuk perhitungan suku bunga. Kami melakukannya secara numerik dan simbolis untuk menunjukkan kepada Anda bagaimana Euler dapat digunakan untuk memecahkan masalah kehidupan nyata.

Asumsikan Anda memiliki modal awal 5000 (katakanlah dalam dolar).

\textgreater K=5000

\begin{verbatim}
5000
\end{verbatim}

Sekarang asumsikan tingkat bunga 3\% per tahun. Mari kita tambahkan satu tarif sederhana dan menghitung hasilnya.

\textgreater K*1.03

\begin{verbatim}
5150
\end{verbatim}

Euler juga akan memahami sintaks berikut.

\textgreater K+K*3\%

\begin{verbatim}
5150
\end{verbatim}

Tetapi lebih mudah untuk menggunakan faktor tersebut

\textgreater q=1+3\%, K*q

\begin{verbatim}
1.03
5150
\end{verbatim}

Selama 10 tahun, kita cukup mengalikan faktor-faktor dan mendapatkan nilai akhir dengan suku bunga majemuk.

\textgreater K*q\^{}10

\begin{verbatim}
6719.58189672
\end{verbatim}

Untuk tujuan kita, kita dapat mengatur format menjadi 2 digit setelah titik desimal.

\textgreater format(12,2); K*q\^{}10

\begin{verbatim}
    6719.58 
\end{verbatim}

Mari kita cetak yang dibulatkan menjadi 2 digit dalam kalimat lengkap.

\textgreater{}``Starting from'' + K + ``\$ you get'' + round(K*q\^{}10,2) + ``\$.''

\begin{verbatim}
Starting from 5000$ you get 6719.58$.
\end{verbatim}

Bagaimana jika kita ingin mengetahui hasil menengah dari tahun 1 hingga tahun 9? Untuk ini, bahasa matriks Euler sangat membantu. Anda tidak perlu menulis loop, tetapi cukup masukkan

\textgreater K*q\^{}(0:10)

\begin{verbatim}
Real 1 x 11 matrix

    5000.00     5150.00     5304.50     5463.64     ...
\end{verbatim}

Bagaimana keajaiban ini terjadi? Pertama, ekspresi 0:10 mengembalikan vektor bilangan bulat.

\textgreater short 0:10

\begin{verbatim}
[0,  1,  2,  3,  4,  5,  6,  7,  8,  9,  10]
\end{verbatim}

Kemudian semua operator dan fungsi di Euler dapat diterapkan pada vektor elemen untuk elemen. Jadi

\textgreater short q\^{}(0:10)

\begin{verbatim}
[1,  1.03,  1.0609,  1.0927,  1.1255,  1.1593,  1.1941,  1.2299,
1.2668,  1.3048,  1.3439]
\end{verbatim}

adalah vektor faktor q\^{}0 hingga q\^{}10. Ini dikalikan dengan K, dan kita mendapatkan vektor nilai.

\textgreater VK=K*q\^{}(0:10);

Of course, the realistic way to compute these interest rates would be to round to the nearest cent after each year. Let us add a function for this.

\textgreater function oneyear (K) := round(K*q,2)

Mari kita bandingkan kedua hasil, dengan dan tanpa pembulatan.

\textgreater longest oneyear(1234.57), longest 1234.57*q

\begin{verbatim}
                1271.61 
              1271.6071 
\end{verbatim}

Sekarang tidak ada rumus sederhana untuk tahun ke-n, dan kita harus mengulangi tahun-tahun. Euler memberikan banyak solusi untuk ini.

Cara termudah adalah fungsi iterate, yang mengulangi fungsi tertentu beberapa kali.

\textgreater VKr=iterate(``oneyear'',5000,10)

\begin{verbatim}
Real 1 x 11 matrix

    5000.00     5150.00     5304.50     5463.64     ...
\end{verbatim}

Kita dapat mencetaknya dengan cara yang ramah, menggunakan format kita dengan tempat desimal tetap.

\textgreater VKr'

\begin{verbatim}
    5000.00 
    5150.00 
    5304.50 
    5463.64 
    5627.55 
    5796.38 
    5970.27 
    6149.38 
    6333.86 
    6523.88 
    6719.60 
\end{verbatim}

Untuk mendapatkan elemen vektor tertentu, kita menggunakan indeks dalam tanda kurung siku.

\textgreater VKr{[}2{]}, VKr{[}1:3{]}

\begin{verbatim}
    5150.00 
    5000.00     5150.00     5304.50 
\end{verbatim}

Anehnya, kita juga bisa menggunakan vektor indeks. Ingatlah bahwa 1:3 menghasilkan vektor {[}1,2,3{]}.

Mari kita bandingkan elemen terakhir dari nilai yang dibulatkan dengan nilai penuh.

\textgreater VKr{[}-1{]}, VK{[}-1{]}

\begin{verbatim}
    6719.60 
    6719.58 
\end{verbatim}

Perbedaannya sangat kecil.

\chapter{Memecahkan Persamaan}\label{memecahkan-persamaan}

Sekarang kita mengambil fungsi yang lebih maju, yang menambahkan tingkat uang tertentu setiap tahun.

\textgreater function onepay (K) := K*q+R

Kita tidak harus menentukan q atau R untuk definisi fungsi. Hanya jika kita menjalankan perintah, kita harus menentukan nilai-nilai ini. Kami memilih R = 200.

\textgreater R=200; iterate(``onepay'',5000,10)

\begin{verbatim}
Real 1 x 11 matrix

    5000.00     5350.00     5710.50     6081.82     ...
\end{verbatim}

Bagaimana jika kita menghapus jumlah yang sama setiap tahun?

\textgreater R=-200; iterate(``onepay'',5000,10)

\begin{verbatim}
Real 1 x 11 matrix

    5000.00     4950.00     4898.50     4845.45     ...
\end{verbatim}

Kami melihat bahwa uang berkurang. Jelas, jika kita hanya mendapatkan 150 bunga di tahun pertama, tetapi menghapus 200, kita kehilangan uang setiap tahun.

Bagaimana kita bisa menentukan berapa tahun uang itu akan bertahan? Kita harus menulis loop untuk ini. Cara termudah adalah dengan mengulangi cukup lama.

\textgreater VKR=iterate(``onepay'',5000,50)

\begin{verbatim}
Real 1 x 51 matrix

    5000.00     4950.00     4898.50     4845.45     ...
\end{verbatim}

Dengan menggunakan bahasa matriks, kita dapat menentukan nilai negatif pertama dengan cara berikut.

\textgreater min(nonzeros(VKR\textless0))

\begin{verbatim}
      48.00 
\end{verbatim}

Alasannya adalah bahwa nonnol(VKR\textless0) mengembalikan vektor indeks i, di mana VKR{[}i{]}\textless0, dan min menghitung indeks minimal.

Karena vektor selalu dimulai dengan indeks 1, jawabannya adalah 47 tahun.

Fungsi iterate() memiliki satu trik lagi. Itu bisa mengambil kondisi akhir sebagai argumen. Kemudian akan mengembalikan nilai dan jumlah iterasi.

\textgreater\{x,n\}=iterate(``onepay'',5000,till=``x\textless0''); x, n,

\begin{verbatim}
     -19.83 
      47.00 
\end{verbatim}

Mari kita coba menjawab pertanyaan yang lebih ambigu. Asumsikan kita tahu bahwa nilainya adalah 0 setelah 50 tahun. Berapa suku bunganya?

Ini adalah pertanyaan, yang hanya bisa dijawab secara numerik. Di bawah ini, kita akan mendapatkan rumus yang diperlukan. Kemudian Anda akan melihat bahwa tidak ada rumus mudah untuk suku bunga. Tapi untuk saat ini, kami bertujuan untuk solusi numerik.

Langkah pertama adalah mendefinisikan fungsi yang melakukan iterasi n kali. Kami menambahkan semua parameter ke fungsi ini.

\textgreater function f(K,R,P,n) := iterate(``x*(1+P/100)+R'',K,n;P,R){[}-1{]}

Iterasinya seperti di atas

lateks: x\_\{n+1\} = x\_n cdot left(1+ frac\{P\}\{100\}right) + R

Tetapi kita lebih lama menggunakan nilai global R dalam ekspresi kita. Fungsi seperti iterate() memiliki trik khusus di Euler. Anda dapat meneruskan nilai variabel dalam ekspresi sebagai parameter titik koma. Dalam hal ini P dan R.

Selain itu, kami hanya tertarik pada nilai terakhir. Jadi kita ambil indeks {[}-1{]}.

Mari kita coba tes.

\textgreater f(5000,-200,3,47)

\begin{verbatim}
     -19.83 
\end{verbatim}

Sekarang kita bisa menyelesaikan masalah kita.

\textgreater solve(``f(5000,-200,x,50)'',3)

\begin{verbatim}
       3.15 
\end{verbatim}

Rutinitas pemecahan memecahkan expression=0 untuk variabel x. Jawabannya adalah 3,15\% per tahun. Kami mengambil nilai awal 3\% untuk algoritma. Fungsi solve() selalu membutuhkan nilai awal.

Kita dapat menggunakan fungsi yang sama untuk memecahkan pertanyaan berikut: Berapa banyak yang dapat kita keluarkan per tahun sehingga modal awal habis setelah 20 tahun dengan asumsi tingkat bunga 3\% per tahun.

\textgreater solve(``f(5000,x,3,20)'',-200)

\begin{verbatim}
    -336.08 
\end{verbatim}

Perhatikan bahwa Anda tidak dapat menyelesaikan jumlah tahun, karena fungsi kita mengasumsikan n sebagai nilai bilangan bulat.

\section{Solusi Simbolis untuk Masalah Suku Bunga}\label{solusi-simbolis-untuk-masalah-suku-bunga}

Kita dapat menggunakan bagian simbolis Euler untuk mempelajari masalah tersebut. Pertama, kita mendefinisikan fungsi kita onepay() secara simbolis.

\textgreater function op(K) \&= K*q+R; \$\&op(K)

Sekarang kita dapat mengulangi ini.

\textgreater\$\&op(op(op(op(K)))), \$\&expand(\%)

Kita melihat pola. Setelah n periode kita memiliki

lateks: K\_n = q\^{}n K + R (1+q+ldots+q\^{}\{n-1\}) = q\^{}n K + frac\{q\^{}n-1\}\{q-1\} R

Rumusnya adalah rumus untuk jumlah geometris, yang diketahui oleh Maxima.

\textgreater\&sum(q\^{}k,k,0,n-1); \$\& \% = ev(\%,simpsum)

Operasi ini cukup rumit. Jumlah dievaluasi dengan bendera ``simpsum'' untuk menguranginya menjadi hasil bagi.

Mari kita buat fungsi untuk ini.

\textgreater function fs(K,R,P,n) \&= (1+P/100)\^{}n*K + ((1+P/100)\^{}n-1)/(P/100)*R; \$\&fs(K,R,P,n)

Fungsi ini melakukan hal yang sama seperti fungsi f sebelumnya. Tapi itu lebih efektif.

\textgreater longest f(5000,-200,3,47), longest fs(5000,-200,3,47)

\begin{verbatim}
     -19.82504734650985 
     -19.82504734652684 
\end{verbatim}

Sekarang kita dapat menggunakannya untuk meminta waktu n.~Kapan modal kita habis? Tebakan awal kami adalah 30 tahun.

\textgreater solve(``fs(5000,-330,3,x)'',30)

\begin{verbatim}
      20.51 
\end{verbatim}

Jawaban ini mengatakan bahwa itu akan menjadi negatif setelah 21 tahun.

Kita juga dapat menggunakan sisi simbolis Euler untuk menghitung rumus pembayaran.

Asumsikan kita mendapatkan pinjaman K, dan membayar n pembayaran R (mulai setelah tahun pertama) meninggalkan sisa hutang Kn (pada saat pembayaran terakhir). Rumus untuk ini jelas

\textgreater equ \&= fs(K,R,P,n)=Kn; \$\&equ

Biasanya rumus ini diberikan dalam hal

lateks: i = frac\{P\}\{100\}

\textgreater equ \&= (equ with P=100*i); \$\&equ

Kita dapat menyelesaikan tarif R secara simbolis.

\textgreater\$\&solve(equ,R)

Seperti yang Anda lihat dari rumus, fungsi ini mengembalikan kesalahan floating point untuk i=0. Euler tetap menjalankannya.

Tentu saja, kami memiliki batasan sebagai berikut.

\textgreater\$\&limit(R(5000,0,x,10),x,0)

Jelas, tanpa bunga kita harus membayar kembali 10 kali cicilan sebesar 500.

Persamaan tersebut juga bisa diselesaikan untuk

?

. Persamaan tersebut akan terlihat lebih baik jika kita menerapkan beberapa penyederhanaan.

\textgreater fn \&= solve(equ,n) \textbar{} ratsimp; \$\&fn

\begin{center}\rule{0.5\linewidth}{0.5pt}\end{center}

\chapter{Menyelesaikan Soal Aljabar}\label{menyelesaikan-soal-aljabar}

\begin{enumerate}
\def\labelenumi{\arabic{enumi}.}
\tightlist
\item
  Sederhanakan soal eksponen ini
\end{enumerate}

\[\frac{x^-5}{y - 4}\]penyelesaian:

\textgreater\$\&x\textsuperscript{-5/y}-4

\[\frac{y^4}{x^5}\] 2. Sederhanakan soal eksponen ini

\[\frac{a^-2}{b^-8}\]penyelesaian:

\textgreater\$\&a\textsuperscript{-2/b}-8

\[\frac{b^8}{a^2}\] 3. Sederhanakan eksponensial berikut:

\[\left( \frac{24a^{10}b^{-8}c^{7}}{12a^{6}b^{-3}c^{5}} \right)^{-5}\]penyelesaian:

\textgreater\$\&((24*a\textsuperscript{10*b}(-8)*c\^{}7) / (12*a\textsuperscript{6*b}(-3)*c\textsuperscript{5))}(-5)

\[\frac{b^{25}}{32\,a^{20}\,c^{10}}\] 4. Sederhanakan bentuk berikut

\[\left( \frac{125p^{12}q^{-14}r^{22}}{25p^{8}q^{-6}r^{-15}} \right)^{-4}\]penyelesaian:

\textgreater\$\& ((125*p\textsuperscript{12*q}-14*r\^{}22) / (25*p\textsuperscript{8*q}6*r\textsuperscript{-15))}(-4)

\[\frac{q^{80}}{625\,p^{16}\,r^{148}}\] 5. Sederhanakan bentuk berikut

\[\frac{4(8 - 6)^2 - 4 \cdot 3 + 2 \cdot 8}{3^1 + 19^0}\]penyelesaian:

\textgreater\&(4*(8-6)\textsuperscript{2-4*3+2*8)/(3}1+19\^{}0)

\begin{verbatim}
                                  5
\end{verbatim}

\begin{enumerate}
\def\labelenumi{\arabic{enumi}.}
\setcounter{enumi}{5}
\tightlist
\item
  Hitung operasi berikut
\end{enumerate}

\[(x + 6)(x + 3)\]penyelesaiannya:

\textgreater\$\&expand((x + 6)*(x + 3))

\[x^2+9\,x+18\] 7. Hitung operasi berikut

\[(2a + 3)(a + 5)\]penyelesaiannya:

\textgreater\&\$expand((2*a + 3)*(a + 5))

\begin{verbatim}
                              2
                           2 a  + 13 a + 15
\end{verbatim}

\begin{enumerate}
\def\labelenumi{\arabic{enumi}.}
\setcounter{enumi}{7}
\tightlist
\item
  hitung operasi berikut
\end{enumerate}

\[(2x + 3y)(2x + y)\]penyelesaiannya:

\textgreater\$\&expand((2*x + 3*y)*(2*x + y))

\[3\,y^2+8\,x\,y+4\,x^2\] 9. Hitung operasi berikut

\[(x + 3)^2\]penyelesaiannya:

\textgreater\$\&expand((x + 3)\^{}2)

\[x^2+6\,x+9\] 10. Hitunglah operasi berikut

\[(y - 5)^2\]

hasilnya:

\textgreater\$\&expand((y - 5)\^{}2)

\[y^2-10\,y+25\] 11. Faktorisasikan

\[t^2 + 8t + 15\]hasilnya:

\textgreater\$\&factor(t\^{}2 - 8*t + 15)

\[\left(t-5\right)\,\left(t-3\right)\] 12. Faktorkan

\[y^2 + 12y + 27\]hasilnya:

\textgreater\$\&factor(y\^{}2 + 12*y + 27)

\[\left(y+3\right)\,\left(y+9\right)\] 13. Faktorkan

\[z^2 - 81\]hasilnya:

\textgreater\$\&factor(z\^{}2 - 81)

\[\left(z-9\right)\,\left(z+9\right)\] 14. Faktor dari

\[(m^2 - 4)\]hasilnya:

\textgreater\$\&factor(m\^{}2 - 4)

\[\left(m-2\right)\,\left(m+2\right)\] 15. Faktor dari

\[(16x^2 - 9)\]hasilnya

\textgreater\$\&factor(16*x\^{}2 - 9)

\[\left(4\,x-3\right)\,\left(4\,x+3\right)\] 16. penyelesaian dari

\[y^2 - 4y - 45 = 0\]hasilnya adalah

\textgreater\$\&solve(y\^{}2 - 4*y - 45 = 0, y)

\[\left[ y=9 , y=-5 \right] \] 17. selesaikan

\[t^2 + 6t = 0\]hasilnya:

\textgreater\$\&solve(t\^{}2 + 6*t = 0, t)

\[\left[ t=-6 , t=0 \right] \]18. carilah solusinya

\[ n^2 + 4n + 4 = 0 \]\textgreater\$\&solve(n\^{}2 + 4*n + 4 = 0, n)

\[\left[ n=-2 \right]\] 19. carilah solusinya

\[y^2 + 25 = 10y\]solusinya:

\textgreater\$\&solve(y\^{}2 + 25 - 10*y = 0, y)

\[\left[ y=5 \right]\] 20. Tunjukkan solusinya

\[y^2-4y-45= 0\]hasilnya:

\textgreater\$\&solve(y\^{}2 - 4*y - 45 = 0, y)

\backmatter
\end{document}
